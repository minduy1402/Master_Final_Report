%% LaTeX2e template for FEUP's Projeto Integrador
%% 
%% FEUP, JCL, Tue Jan 30 21:57:48 2024
%%
%% PLEASE send improvements to jlopes at fe.up.pt (up till 2028 :-)

\documentclass[11pt,a4paper]{report}

%%------------------------------- preamble ------------------------------------

%% comment next for EN
\usepackage[english]{babel}      % language PT 
\usepackage[utf8]{inputenc}       % accents
\usepackage[T1]{fontenc}          % PS fonts
\usepackage{newtxtext,newtxmath}  % do not use CM fonts
\usepackage{amsmath}              % multi-line and other mathematical statements
\usepackage{setspace}             % setting the spacing between lines
\usepackage{graphicx}             % go far beyond what the graphics package
\usepackage[normalem]{ulem}       % various types of underlining
\usepackage{caption}              % rotating captions, sideways captions, etc.
\usepackage{float}                % tables and figures in the multi-column environment 
\usepackage{subcaption}           % for subfigures and the like
\usepackage{longtable}            % tables that continue to the next page
\usepackage{multirow}             % tabular cells spanning multiple rows
\usepackage[table]{xcolor}        % driver-independent color extensions
\usepackage{lipsum}               % loren dummy text
\setlength{\marginparwidth}{2cm}  % todonotes' requirements
\usepackage{todonotes}            % todo's
\usepackage{csquotes}             % context sensitive quotation facilities
\usepackage[backend=biber,authordate]{biblatex-chicago}  % Chicago Manual of Style
\usepackage{pgfgantt}             % Gantt charts

%% document dimensions
\usepackage[a4paper,left=25mm,right=25mm,top=25mm,bottom=25mm,headheight=6mm,footskip=12mm]{geometry}
\setlength{\parindent}{0em}
\setlength{\parskip}{1ex}

%% headers & footers
\usepackage{lastpage}
\usepackage{fancyhdr}
\fancyhf{}                            % clear off all default fancyhdr headers and footers
\rhead{\small{\emph{\projtitle, \projauthor}}}
\rfoot{\small{\thepage\ / \pageref{LastPage}}}
\pagestyle{fancy}                     % apply the fancy header style
\renewcommand{\headrulewidth}{0.4pt}
\renewcommand{\footrulewidth}{0.4pt}

%% colors
\usepackage{color}
\definecolor{engineering}{rgb}{0.549,0.176,0.098}
\definecolor{cloudwhite}{cmyk}{0,0,0,0.025}

%% source-code listings
\usepackage{listings}
\lstset{ %
 language=C,                        % choose the language of the code
 basicstyle=\footnotesize\ttfamily,
 keywordstyle=\bfseries,
 numbers=left,                      % where to put the line-numbers
 numberstyle=\scriptsize\texttt,    % the size of the fonts that are used for the line-numbers
 stepnumber=1,                      % the step between two line-numbers. If it's 1 each line will be numbered
 numbersep=8pt,                     % how far the line-numbers are from the code
 frame=tb,
 float=htb,
 aboveskip=8mm,
 belowskip=4mm,
 backgroundcolor=\color{cloudwhite},
 showspaces=false,                  % show spaces adding particular underscores
 showstringspaces=false,            % underline spaces within strings
 showtabs=false,                    % show tabs within strings adding particular underscores
 tabsize=2,                         % sets default tabsize to 2 spaces
 captionpos=t,                      % sets the caption-position to top
 belowcaptionskip=12pt,             % space between caption and listing
 breaklines=true,                   % sets automatic line breaking
 breakatwhitespace=false,           % sets if automatic breaks should only happen at whitespace
 escapeinside={\%*}{*)},            % if you want to add a comment within your code
 morekeywords={*,var,template,new}  % if you want to add more keywords to the set
}

%% hyperreferences (HREF, URL)
\usepackage{hyperref}
\hypersetup{
    plainpages=false, 
    pdfpagelayout=SinglePage,
    bookmarksopen=false,
    bookmarksnumbered=true,
    breaklinks=true,
    linktocpage,
    colorlinks=true,
    linkcolor=engineering,
    urlcolor=engineering,
    filecolor=engineering,
    citecolor=engineering,
    allcolors=engineering
}

%% path to the figures directory
\graphicspath{{figures/}}

%% bibliography file, must be in preamble
\addbibresource{bibliography.bib}

%% macros, to be updated as needed
\newcommand{\school}{Institut National des Sciences Appliquées de Toulouse }
\newcommand{\degree}{Bachelor's Degree in Computer Science and Engineering}
\newcommand{\projtitle}{Optimizing and Adapting Language Models for Domain-Specific Task}
\newcommand{\subtitle}{End-of-studies Apprenticeship Report}
\newcommand{\projauthor}{Minh Duy Nguyen}
\newcommand{\supervisor}{Milad Mozafari (Torus AI), David Bertoin (INSA Toulouse)}
\newcommand{\tutor}{David Bertoin}

%% my other macros, if needed
\newcommand{\windspt}{\textsf{WindsPT\/}}
\newcommand{\windscannerpt}{\emph{Windscanner.PT\/}}
\newcommand{\class}[1]{{\normalfont\sffamily #1\/}}
\newcommand{\svg}{\class{SVG}}

%% my environments for infos
\newenvironment{info}[1]{\vspace*{6mm}\color{blue}[ \textbf{INFO:} \begin{em} #1}
                        {\vspace*{3mm}\end{em} ]}
\newenvironment{infoopt}[1]{\vspace*{6mm}\color{blue}[ \textbf{INFO (optional element):} \begin{em} #1}
                        {\vspace*{3mm}\end{em} ]}

%%------------------------------- document-------------------------------------

\begin{document}

%% preamble page numbers with roman numerals
\pagenumbering{roman}\setcounter{page}{1}
\pagestyle{plain}

%%------------------------------- cover page ----------------------------------

\begin{titlepage}
\center

\vspace{-15mm}
{\large \textbf{\textsc{\school}}}\\

\vfill

{\Large \textbf{\projtitle}}\\[8mm]
{\large \textbf{\subtitle}}\\[28mm]

{\Large \textbf{\projauthor}}\\

\vfill

\includegraphics[width=52mm]{figures/logo-INSA.png}
\includegraphics[width=52mm]{figures/logo-N7.png}

\vfill

{\large \degree}\\[8mm]
% {\large \textbf{Tutor na U.Porto}: \tutor}\\[2mm]
{\large \textbf{Supervised by}: \supervisor}\\[8mm]

%\renewcommand{\today}{15 de dezembro de 2023}
\today

\end{titlepage}

%%------------------------------- Abstract ------------------------------------

\chapter*{Abstract}

\begin{info}
The abstract is essentially informative in nature and should be
written concisely (up to 200 words) in a way that captures the
interest of the reader.

The Abstract replaces reading the document and does not contain figures,
tables, citations, etc.\ 
It should include the following topics: scope, objectives, methods,
main findings, including results, conclusions and
recommendations, if any.

For more information on how to write a good abstract, consult the online
tutorial available on the Library website, ``Publication Support Guide'',
section: 
``\href{https://docs.google.com/document/d/1TDC1behVq8x7fQL4CcPEEh_np5GXviJevQxnQ9gbiJs/edit\#heading=h.s4z9k57ywd9w}
{Structuring Technical Report}''.
\end{info}

\todo[inline]{Write the Abstract, but only at the end.}

%\vspace{\fill}
%{\Large \textbf{Palavras-chave}:} palavra1, palavra2, palavra3, palavra4
%\vspace*{24mm}

%%------------------------------- Acknowledgments -----------------------------

\chapter*{Acknowledgement}

\begin{infoopt}
Usually the contribution of other people or entities is mentioned,
both for carrying out the study and for producing the report.
They can be done on a separate page or included in the introduction.
\end{infoopt}

%%------------------------------- table of contents ---------------------------

%% redefine tableofcontents text, ONLY for PT
\renewcommand{\contentsname}{Table of contents}

\tableofcontents

%%------------------------------- list of todos -------------------------------

% list todos; comment in the end (should be empty before delivery :-)
\listoftodos

\begin{infoopt}
Annotations can be placed during document preparation, which
are listed here.

This element does not appear in the final document!
\end{infoopt}

%%------------------------------- Glossary ------------------------------------

\chapter*{Glossary}
%\addcontentsline{toc}{chapter}{Glossary}

\begin{description}
\item[bash] \hfill \\
  Bash is a \emph{Unix shell} and command language written
  in 1989 by Brian Fox for the GNU Project as a free
  software replacement for the \emph{Bourne shell}.
\item[firewall] \hfill \\
  In computing, a \emph{firewall} is a network security system
  that monitors and controls incoming and outgoing network traffic
  based on predetermined security rules.
  A \emph{firewall} typically establishes a barrier between a
  trusted network and an untrusted network, such as the Internet.
\item[Glossary] \hfill \\
  A glossary is a kind of small specific dictionary for
  little-known words and expressions present in a text, whether
  they are technical, regional or from another language.
\end{description}

\begin{infoopt}
Having a glossary is justified whenever it is necessary to clarify
to the reader the meaning of specific terminology used in the report text.
Its location in the initial elements is recommended, although in
existing standards there are variants, and it can also appear in
the final elements.

The list of items should be sorted alphabetically\footnote{%
For more information, consult the online tutorial
``\href{https://docs.google.com/document/d/1TDC1behVq8x7fQL4CcPEEh_np5GXviJevQxnQ9gbiJs/edit}
{Publication Support Guide}''.}.
\end{infoopt}

%%------------------------------- chapter ------------------------------------

\chapter{Introduction}

%% display headers & footers
\pagestyle{fancy}
%% main page numbers with arabic numerals
\pagenumbering{arabic}\setcounter{page}{1}

\begin{info}
Succinct contextualization of the report subject, making
reference to the scope and objectives.
Here the motivation of the presented work is clarified and the
adopted approach and its relationship with similar works is explained,
from a generic perspective.
Details about what is explained in the following sections
should not be anticipated.
If relevant, the target audience can also be indicated\footnote{%
For more information, consult the online tutorial
``\href{https://docs.google.com/document/d/1TDC1behVq8x7fQL4CcPEEh_np5GXviJevQxnQ9gbiJs/edit}
{Publication Support Guide}''.}.
\end{info}

This report addresses \ldots\ provides an overview of the context, objectives and approach adopted.

\section{Context}

\begin{info}
   Present the organizational context in which the project/internship took place (company, institution, research unit, research laboratory, etc.).
   Present the problem addressed and the motivation for the work carried out (what is the problem addressed and why is it important).
\end{info}

\section{Objectives and expected results}

\begin{info}
  Indicate the objectives of the work and the expected results.
\end{info}

\section{Report structure}
	
\begin{info}
  Briefly describe the structure of the report.

  It is expected that the report will have between 10 to 25 pages (in A4 format, single column, with a font size not exceeding 12pt in paragraph text), already including appendices.
\end{info}

In addition to the introduction, this report is organized into 4 more chapters.
In Chapter~\ref{chap:metodo} \ldots

%%------------------------------- DELETEME ----------------------------------

\newpage  % don't do this
\section*{Examples}

\todo[inline]{Remove the ``Examples'' section when no longer needed.}

\begin{info}
  Some parts of the document are illustrated below.

  This section does not appear in the final document!
\end{info}

\subsection*{Equations}

\begin{info}
This text is just an example that precedes an equation.
\end{info}  

Simple equations can be inserted inline with the text:
the line is \(y=mx+b\).

More complicated equations should be separated into individual lines and
numbered sequentially on the right within parentheses.
This is the generic quadratic equation:

\begin{equation} \label{eq:1}
  ax^2+bx+cx=0
\end{equation}

Where $a$ is the quadratic coefficient; $b$ the linear coefficient; $c$ the
constant coefficient independent of variable $x$, to be determined.

Equations should be referred to by their number.
For example, Equation~\ref{eq:2} solves problems formulated as
shown in Equation~\ref{eq:1}.

\begin{equation} \label{eq:2}
  x=\frac{-b\pm \sqrt{b^2-4ac}}{2a}
\end{equation}

\subsection*{Figures and tables}

All figures and tables must be captioned and
numbered sequentially:

\begin{itemize}
\item figures should be captioned below;
\item tables should be captioned at the top.
\end{itemize}

Keep figures centered and inline with text so that the
caption always appears attached to the image.

\begin{info}
Figures should float freely on the page and be referenced and
described in the text, with sources properly stated,
to avoid plagiarism.
\end{info}

As an example, Figure~\ref{fig:campus} % (taken from \url{www.fe.up.pt})
shows the FEUP \emph{campus}. 

\lipsum[1]

\begin{figure}
\centering
\includegraphics[width=0.8\textwidth]{campus}
\caption[Aerial photograph of the FEUP Campus]{Aerial photograph of the FEUP Campus~\cite{kn:figura}.} \label{fig:campus}
\end{figure}

\lipsum[2]

\begin{info}
Space can be reserved for placing a figure in the future; for
example, Figure~\ref{fig:natal}.
\end{info}

\lipsum[3]

\begin{figure} [b]
  \centering
  \missingfigure{Insert the Christmas figure.}
  \caption{Christmas at the FEUP Campus.} \label{fig:natal}
\end{figure}

\lipsum[4]

\begin{info}
Tables should float freely on the page and be referenced and
described in the text, with sources properly stated,
to avoid plagiarism.
\end{info}

Table~\ref{tab:feup} % (excerpt adapted from ``FEUP in numbers'', 2011)
serves to exemplify how to show some values that, in this case, are
related to some numerical data associated with resources and investments
of FEUP in the year 2011.

\lipsum[4]

\begin{table}
  \centering
  \caption[FEUP Physical Resources]{FEUP Physical Resources. Adapted from~\cite{kn:tabela}}
  \begin{tabular}{| l | c |}
    \hline
    \textbf{Description} & \textbf{Quantity}\\\hline
    \hline
    Total area of FEUP campus & 93 918 $m^2$\\\hline
    Green spaces & 23 000 $m^2$\\\hline
    Number of computers dedicated to teaching & 1 815\\\hline
    Investment in laboratory equipment & 1.46 M€\\
    \hline
  \end{tabular}
  \label{tab:feup}
\end{table}

\lipsum[6]

\subsection*{Citations}

As you write the report text, you should indicate the works
of other authors on which you are based, in the form of citations.
This consists of briefly indicating the sources used from which
you obtained additional information to develop the topic of your
report.

There are two main ways to cite:
\begin{itemize}
\item by \textbf{paraphrase}: interpretation of the original content in
  words different from those of the consulted source, indicating the source
  immediately after; or
\item
  by \textbf{transcription}: use of an excerpt of the original content
  presenting it in quotation marks, indicating the source immediately after.
\end{itemize}

Citations must follow a standardized style.
Among the many that exist, the FEUP Library recommends the
Chicago style (author-date format).

\begin{info}
Below are some examples, at random, of citations (by paraphrase)
according to this style.
\end{info}

The decision to choose a topic for an academic work can
vary~\cite{kn:Bel02-book}.
The topic can be thought of and chosen by the student themselves, or from
a list of topics already conceived, with potential interest
for study~\cite{kn:GLPR14-joPhysics}.

Each citation throughout the text should correspond to a reference
indicated in the final list of bibliographic
references~\cite{kn:Lip08,kn:MSS+12-wemep,kn:VKL+18-dtu}.

It is important not to forget that figures (images, tables,
graphs, etc.) from works by other authors (for example
obtained through the Internet) should always be cited, after the
respective captions~\cite{kn:GLPC22-torque}.

For more information on this subject and to see examples, consult the guide
``Avoiding plagiarism: good practices in the use of
information''\footnote{\url{https://feup.libguides.com/plagio/citar}}.  

Duis non odio morbi ut dui sed accumsan risus eget odio~\cite{iso_19156_2011,ornelas2016}. 

%\lipsum[7]

\subsection*{Code}

\begin{info}
Below is an illustration of including code in the document.
\end{info}

Listing~\ref{code:useless} 
\lipsum[8]

\begin{lstlisting}[language=Python, caption=Python example, label=code:useless]
# Take the user's input
words = input("Enter the text to translate to pig latin: ")
print(f"You entered: {words}")

# Break apart the words into a list
words = words.split(' ')

# Use a list comprehension to translate words greater than or equal to 3 characters
translated_words = [(w[1:] + w[0] + "ay") for w in words if len(w) >= 3 ]

# Print each translated word
for word in translated_words:
    print(word)
\end{lstlisting}

\subsection*{Using macros}

\begin{info}
Below is an illustration of using \LaTeX{} macros defined in
the preamble.
Note the use of the \verb!\class{}! macro for classes, methods and components.
\end{info}

The \windspt, taken from \windscannerpt, uses \svg\ \ldots\ as in \class{Student.calculate-age()}.

\lipsum[9]

\begin{info}
The subsequent component parts that make up the body of the text
should be structured in sections, estimating that up to 3 levels is
sufficient for this type of work.

For more information, consult the online tutorial
``\href{https://docs.google.com/document/d/1TDC1behVq8x7fQL4CcPEEh_np5GXviJevQxnQ9gbiJs/edit}
{Publication Support Guide}''.
Note that the sections indicated there can be adapted according to the topic
or depth of the study to be developed.
\end{info}

\begin{info}
It is not customary to have consecutive section headings without text.
\end{info}

\subsection*{The dash}

About the use of the hyphen and the dash\footnote{``Portuguese Language Questions'', \url{https://ciberduvidas.iscte-iul.pt/consultorio/perguntas/o-uso-do-hifen-e-do-travessao/31251}}:
\begin{enumerate}
    \item The hyphen (Alt + 0045): without spaces (-);
    \item The em-dash (Alt + 0151): in Portuguese, surrounded by spaces (---);
    \item The en-dash (Alt + 150): without spaces (--);
    \item The mathematical subtraction sign: without spaces (–).
\end{enumerate}

\subsection*{Quotation marks}

About quotation marks in \LaTeX, either the glyph is used directly, or they are made with the backtick at the beginning and the apostrophe at the end, as in ``example''\footnote{What is the best way to use quotation mark glyphs?
\url{https://tex.stackexchange.com/questions/531/what-is-the-best-way-to-use-quotation-mark-glyphs}}.

%%------------------------------- chapter ------------------------------------

\chapter{Methodology used and main activities developed}\label{chap:metodo}

In this chapter, the methodology followed is described, the main
participants in the project are listed and the main
activities developed are recorded.

\section{Methodology used}

\begin{info}
Describe the methodology~\cite{despa2014comparative} followed (for
example, iterative development with biweekly \emph{sprints} and weekly
follow-up meetings) and the resources used (for example,
GitHub\footnote{\url{https://github.com/}}, etc.).
\end{info}

\section{Participants, roles and responsibilities}

\begin{info}
Identify the project team, \emph{stakeholders} and other participants with whom there was interaction; in the case of group work, clarify the roles and responsibilities of each group member.
\end{info}

\section{Activities developed}

\begin{info}
Describe the activities carried out over time (including relevant events, such as presentations, meetings with clients, etc.) and respective \emph{deliverables}, typically using a Gantt chart~\cite{gantt} (see Figure~\ref{fig:gantt}) and a brief description of each activity/\emph{deliverable}.
It can also be presented through a table with weekly progress.
\end{info}

\begin{figure}
    \begin{ganttchart}[vgrid, hgrid, title height=1,
        x unit=6mm, y unit title=8mm, y unit chart=7mm]{1}{21}

        \gantttitle{Feb.}{4}
        \gantttitle{Mar.}{4}
        \gantttitle{Apr.}{5}
        \gantttitle{May.}{4}
        \gantttitle{Jun.}{4} \\

        \gantttitle{}{1}
        \gantttitle{S1}{1}
        \gantttitle{S2}{1}
        \gantttitle{S3}{1}
        \gantttitle{S4}{1}
        \gantttitle{S5}{1}
        \gantttitle{S6}{1}
        \gantttitle{S7}{1}
        \gantttitle{P}{1}
        \gantttitle{S8}{1}
        \gantttitle{S9}{1}
        \gantttitle{S10}{1}
        \gantttitle{S11}{1}
        \gantttitle{Q}{1}
        \gantttitle{S12}{1}
        \gantttitle{S13}{1}
        \gantttitle{S14}{1}
        \gantttitle{S15}{1}
        \gantttitle{S16}{1}
        \gantttitle{S17}{1}
        \gantttitle{S18}{1} \\

        \ganttbar{\textbf{Kick-off meeting}}{1}{1} \\
        \ganttgroup{Work at the company}{2}{17} \\
        \ganttbar{\textbf{Intermediate meeting}}{11}{11} \\
        \ganttbar{\textbf{Final meeting}}{18}{18} \\
        \ganttbar{Report writing}{18}{20} \\
        \ganttbar{Deliveries}{21}{21}
    \end{ganttchart}
    \caption{Gantt chart.} \label{fig:gantt}
\end{figure}

\chapter{Solution development}

In this chapter, the work developed to achieve the
expected results is described.

If it is the case of a software prototype, the
requirements, solution architecture, development and validation
of the prototype are presented.

\section{Requirements}

\begin{info}
Identify relevant functional and non-functional requirements and respective sources, as well as project constraints.

Below is an example of a table with \textit{user stories}, which can be used if the number of rows is large.
An additional column for priorities may be needed.
\end{info}

\subsection*{User stories}

Examples of possible usage scenarios are presented in Table~\ref{tab:userstories}, using \textit{user stories} written according to the model:
``\textbf{As a <type of user>, I want <some goal> so that <some reason>}''.

\begin{center}
\begin{longtable}{| c | p{18mm} | p{110mm} |}
\caption{Usage scenario} \label{tab:userstories} \\[-2mm] \hline

\textbf{Identifier} & \textbf{Name} & \textbf{Description} \\ \hline\hline
\endfirsthead

\multicolumn{3}{c}{{\tablename\ \thetable{} -- continued from previous page}} \\[2mm] \hline
\textbf{Identifier} & \textbf{Name} & \textbf{Description} \\ \hline\hline
\endhead

\multicolumn{3}{r}{{Continues on next page\ldots}} \\
\endfoot

\hline
\endlastfoot

US01 & Register (High) & As a \textit{Visitor}, I want to register a new account so that I have access to profile data  \\ \hline

% US02 & Login (High) & As a \textit{Visitor}, I want to login so that % I have access to profile data  \\ \hline
% 
% US03 & Edit profile (High) & As a \textit{Authenticated user}, I want % to edit \ldots\  \\ \hline

\end{longtable}
\end{center}

\section{Architecture and technologies (or Design and Implementation)}

\begin{info}
Architecture and technologies used and respective justification (taking into account requirements and existing alternatives), technical diagrams prepared (see Figure~\ref{fig:componentes} taken from SPARX\footnote{\url{https://sparxsystems.com/resources/tutorials/uml2/component-diagram.html}}), technical difficulties encountered and their resolution, etc.
\end{info}

\begin{figure}
\centering
\includegraphics[width=0.72\textwidth]{components}
\caption{UML Component Diagram.} \label{fig:componentes}
\end{figure}

\section{Developed solution}

\begin{info}
Present the developed solution from the user's perspective, with the help of \emph{screenshots}.
\end{info}

\section{Validation}

\begin{info}
Description of the validation of the developed solution, in relation to the identified requirements and constraints, and respective results (for example, experimental evaluation results, tests performed, \emph{feedback} from users or specialists, etc.).
\end{info}

\chapter{Conclusions}

In this chapter, the results achieved and lessons
learned are summarized.
Finally, the limitations of the work are presented and
improvements and future work are proposed.

\section{Results achieved}

\begin{info}
Summarize the results achieved and contributions (in relation to the objectives).

In the case of group work, clarify individual contributions, in qualitative and quantitative terms (percentage).
\end{info}

\lipsum[11]

\section{Lessons learned}

\begin{info}
Reflect on the lessons learned (taking into account the learning objectives).
\end{info}

\lipsum[12]

\section{Future work}

\begin{info}
Identify limitations of the work performed and ideas for improvements and future work.
\end{info}


%%------------------------------- Bibliography --------------------------------

%Here should be the bibliographic references contained in the report text and any relevant bibliography consulted during the project.

\renewcommand{\bibname}{Bibliographic references}
\printbibliography
\addcontentsline{toc}{chapter}{\refname}  % add it to table of contents

\begin{info}
In the final list of references, the works of the authors
cited in abbreviated form throughout the text should appear, obtained automatically
with \class{BibTeX}.
The bibliographic reference is the most developed way of indicating the
sources of information on which it was based.
\end{info}

%%------------------------------- appendix ------------------------------------

\appendix
\chapter{An Appendix}

\begin{infoopt}
Appendices and annexes contain information that complements, supports and
clarifies the report and whose inclusion in the main body of the report
would interfere with a good order of presentation of ideas.

There is an important difference between appendices and annexes: ``In the appendix
only documents authored by the author of the
report are compiled, while in the annex documents authored by
other authors than the report author are compiled.''\footnote{%
For more information, consult the online tutorial
``\href{https://docs.google.com/document/d/1TDC1behVq8x7fQL4CcPEEh_np5GXviJevQxnQ9gbiJs/edit}
{Publication Support Guide}''.}
\end{infoopt}

%%------------------------------- the end. ------------------------------------

\end{document}
