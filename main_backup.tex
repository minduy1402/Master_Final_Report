\documentclass[11pt,a4paper]{report}

%%------------------------------- preamble ------------------------------------

%% comment next for EN
\usepackage[utf8]{inputenc}       % accents
\usepackage[T1]{fontenc}          % PS fonts
\usepackage{newtxtext,newtxmath}  % do not use CM fonts
\usepackage{amsmath}              % multi-line and other mathematical statements
\usepackage{setspace}             % setting the spacing between lines
\usepackage{graphicx}             % go far beyond what the graphics package
\usepackage[normalem]{ulem}       % various types of underlining
\usepackage{caption}              % rotating captions, sideways captions, etc.
\usepackage{float}                % tables and figures in the multi-column environment 
\usepackage{subcaption}           % for subfigures and the like
\usepackage{longtable}            % tables that continue to the next page
\usepackage{multirow}             % tabular cells spanning multiple rows
% \usepackage[table]{xcolor}        % driver-independent color extensions
\usepackage{lipsum}               % loren dummy text
\setlength{\marginparwidth}{2cm}  % todonotes' requirements
\usepackage{todonotes}            % todo's
\usepackage{csquotes}             % context sensitive quotation facilities
\usepackage[backend=biber,authordate]{biblatex-chicago}  % Chicago Manual of Style
\usepackage{pgfgantt}             % Gantt charts
\usepackage[T5]{fontenc}
\usepackage[utf8]{inputenc}
\usepackage[english, vietnamese]{babel}

%% document dimensions
\usepackage[a4paper,left=25mm,right=25mm,top=25mm,bottom=25mm,headheight=6mm,footskip=12mm]{geometry}
\setlength{\parindent}{0em}
\setlength{\parskip}{1ex}

%% headers & footers
\usepackage{lastpage}
\usepackage{fancyhdr}
\fancyhf{}                            % clear off all default fancyhdr headers and footers
\rhead{\small{\emph{\projtitle, \projauthor}}}
\rfoot{\small{\thepage\ / \pageref{LastPage}}}
\pagestyle{fancy}                     % apply the fancy header style
\renewcommand{\headrulewidth}{0.4pt}
\renewcommand{\footrulewidth}{0.4pt}

%% colors
\usepackage{color}
\definecolor{engineering}{rgb}{0.549,0.176,0.098}
\definecolor{cloudwhite}{cmyk}{0,0,0,0.025}

%% source-code listings
\usepackage{listings}
\lstset{ %
 language=C,                        % choose the language of the code
 basicstyle=\footnotesize\ttfamily,
 keywordstyle=\bfseries,
 numbers=left,                      % where to put the line-numbers
 numberstyle=\scriptsize\texttt,    % the size of the fonts that are used for the line-numbers
 stepnumber=1,                      % the step between two line-numbers. If it's 1 each line will be numbered
 numbersep=8pt,                     % how far the line-numbers are from the code
 frame=tb,
 float=htb,
 aboveskip=8mm,
 belowskip=4mm,
 backgroundcolor=\color{cloudwhite},
 showspaces=false,                  % show spaces adding particular underscores
 showstringspaces=false,            % underline spaces within strings
 showtabs=false,                    % show tabs within strings adding particular underscores
 tabsize=2,                         % sets default tabsize to 2 spaces
 captionpos=t,                      % sets the caption-position to top
 belowcaptionskip=12pt,             % space between caption and listing
 breaklines=true,                   % sets automatic line breaking
 breakatwhitespace=false,           % sets if automatic breaks should only happen at whitespace
 escapeinside={\%*}{*)},            % if you want to add a comment within your code
 morekeywords={*,var,template,new}  % if you want to add more keywords to the set
}

%% hyperreferences (HREF, URL)
\usepackage{hyperref}
\hypersetup{
    plainpages=false, 
    pdfpagelayout=SinglePage,
    bookmarksopen=false,
    bookmarksnumbered=true,
    breaklinks=true,
    linktocpage,
    colorlinks=true,
    linkcolor=engineering,
    urlcolor=engineering,
    filecolor=engineering,
    citecolor=engineering,
    allcolors=engineering
}

%% path to the figures directory
\graphicspath{{figures/}}

%% bibliography file, must be in preamble
\addbibresource{bibliography.bib}

%% macros, to be updated as needed
\newcommand{\school}{Institut National des Sciences Appliquées de Toulouse }
\newcommand{\degree}{Bachelor's Degree in Computer Science and Engineering}
\newcommand{\projtitle}{Optimizing and Adapting Language Models for Domain-Specific Task}
\newcommand{\subtitle}{End-of-studies Apprenticeship Report}
\newcommand{\projauthor}{Minh Duy Nguyen}
\newcommand{\supervisor}{Milad Mozafari (Torus AI), David Bertoin (INSA Toulouse)}
\newcommand{\tutor}{David Bertoin}

%% my other macros, if needed
\newcommand{\windspt}{\textsf{WindsPT\/}}
\newcommand{\windscannerpt}{\emph{Windscanner.PT\/}}
\newcommand{\class}[1]{{\normalfont\sffamily #1\/}}
\newcommand{\svg}{\class{SVG}}

%% my environments for infos
\newenvironment{info}[1]{\vspace*{6mm}\color{blue}[ \textbf{INFO:} \begin{em} #1}
                        {\vspace*{3mm}\end{em} ]}
\newenvironment{infoopt}[1]{\vspace*{6mm}\color{blue}[ \textbf{INFO (optional element):} \begin{em} #1}
                        {\vspace*{3mm}\end{em} ]}

%%------------------------------- document-------------------------------------

\begin{document}

%% preamble page numbers with roman numerals
\pagenumbering{roman}\setcounter{page}{1}
\pagestyle{plain}

%%------------------------------- cover page ----------------------------------

\begin{titlepage}
\center

\vspace{-15mm}
{\large \textbf{\textsc{\school}}}\\

\vfill

{\Large \textbf{\projtitle}}\\[8mm]
{\large \textbf{\subtitle}}\\[28mm]

{\Large \textbf{\projauthor}}\\

\vfill

\includegraphics[width=52mm]{figures/logo-INSA.png}
\includegraphics[width=52mm]{figures/logo-N7.png}

\vfill

{\large \degree}\\[8mm]
% {\large \textbf{Tutor na U.Porto}: \tutor}\\[2mm]
{\large \textbf{Supervised by}: \supervisor}\\[8mm]

%\renewcommand{\today}{15 de dezembro de 2023}
\today

\end{titlepage}

%%------------------------------- chapter ------------------------------------
\pagenumbering{arabic}
\chapter{Introduction}

\section{General Context}

\subsection{Sự phát triển của Mô hình Ngôn ngữ Lớn}

Trong thập kỷ qua, lĩnh vực Trí tuệ nhân tạo (Artificial Intelligence - AI) và Xử lý ngôn ngữ tự nhiên (Natural Language Processing - NLP) đã chứng kiến những bước tiến đột phá, đặc biệt kể từ sự ra đời của kiến trúc Transformer [Vaswani et al., 2017]. Các Mô hình Ngôn ngữ Lớn (Large Language Models - LLMs) như GPT-4 [OpenAI, 2023], Claude 3 [Anthropic, 2024], Gemini 1.5 [Google, 2024], cùng các mô hình mã nguồn mở như Llama 3 [Meta, 2024] và Qwen 2.5 [Alibaba, 2024] đã chứng minh khả năng vượt trội trong việc hiểu ngữ cảnh, sinh văn bản và thực hiện suy luận logic trên đa dạng tác vụ.

Theo báo cáo của McKinsey Global Institute (2024), thị trường AI tạo sinh (Generative AI) dự kiến đạt giá trị 4.4 nghìn tỷ USD vào năm 2030. Tuy nhiên, một khảo sát từ Gartner (2024) chỉ ra rằng **67\% doanh nghiệp** gặp khó khăn trong việc triển khai LLMs cho các tác vụ chuyên biệt do các hạn chế về độ chính xác và chi phí vận hành.

\subsection{Khoảng cách giữa Nghiên cứu và Ứng dụng Công nghiệp}

Việc chuyển giao các mô hình ngôn ngữ từ môi trường nghiên cứu sang ứng dụng thực tiễn trong công nghiệp (Industrial Deployment) đang đối mặt với những thách thức đáng kể. Mặc dù các mô hình nền tảng (Foundation Models) sở hữu lượng tri thức tổng quát khổng lồ, chúng thường gặp hai nhóm hạn chế nghiêm trọng khi giải quyết các bài toán chuyên biệt (Domain-specific tasks):

**Khoảng cách về Tri thức (Knowledge Gap):**

Trong các lĩnh vực quan trọng như y tế, tài chính và bảo hiểm, thông tin thường nằm trong các tài liệu phức tạp (báo cáo tài chính, bảng danh mục kỹ thuật y tế CCAM) và thay đổi liên tục. Các LLMs đối mặt với ba vấn đề chính:

- **Dữ liệu tĩnh (Static Knowledge):** Tri thức của mô hình bị giới hạn bởi thời điểm huấn luyện (knowledge cutoff), không thể cập nhật thông tin mới mà không cần huấn luyện lại [Lewis et al., 2020].
- **Thiếu dữ liệu nội bộ (Private Data Inaccessibility):** Mô hình không thể truy cập các tài liệu bảo mật của tổ chức như hợp đồng, quy trình nội bộ, hay hồ sơ bệnh nhân.
- **Hiện tượng ảo giác (Hallucination):** LLMs có xu hướng sinh ra thông tin sai lệch một cách tự tin [Ji et al., 2023], điều không thể chấp nhận trong các quyết định y khoa hay tài chính, nơi yêu cầu độ chính xác gần như tuyệt đối.

**Khoảng cách về Hiệu quả và Hành vi (Efficiency & Behavior Gap):**

Đối với các tác vụ yêu cầu mô hình tuân thủ một kịch bản hành vi cụ thể, một phong cách ngôn ngữ đặc thù (ví dụ: tư vấn viên tâm lý, chuyên gia phân tích), hoặc triển khai trên hạ tầng phần cứng giới hạn, việc sử dụng các mô hình khổng lồ (hàng trăm tỷ tham số) là không tối ưu về chi phí và độ trễ. Theo Hoffmann et al. (2022), chi phí inference của GPT-4 có thể lên tới **$0.06/1000 tokens**, khiến việc triển khai quy mô lớn trở nên không khả thi cho nhiều doanh nghiệp vừa và nhỏ.

\subsection{1.1.3. Động lực nghiên cứu

Xuất phát từ thực tế trên, luận văn này tập trung nghiên cứu và triển khai các kỹ thuật tiên tiến nhằm **tối ưu hóa và thích ứng LLMs cho các miền dữ liệu đặc thù**, với hai hướng tiếp cận chính:

1. **Retrieval-Augmented Generation (RAG):** Tích hợp tri thức bên ngoài để giải quyết Knowledge Gap.
2. **Parameter-Efficient Fine-Tuning (PEFT):** Thích ứng hành vi mô hình để giải quyết Behavior Gap.

\section{Internship Environment}

This master's thesis was completed at Torus AI, a technology company with the mission "Intelligence for Life," specializing in providing advanced AI solutions to improve quality of life and business efficiency.

I worked in the Research and Development Team (R\&D Team - Torus Lab) as a Machine Learning Engineer Alternant. At Torus AI, the R\&D department acts as a bridge between the latest academic research (State-of-the-Art) and commercial products. The team's main task is to continuously explore emerging Generative AI technologies, assess their feasibility, and build functional prototypes (PoCs) to verify their effectiveness before integration into the main product system.

The R\&D work environment demands flexible thinking: not just using existing APIs, but delving into customizing architecture, optimizing data processing pipelines, and quantitatively evaluating technical solutions.

**Vai trò của tác giả:** Kỹ sư Học máy tập sự (Machine Learning Engineer Alternant) trong đội ngũ Nghiên cứu và Phát triển (R&D Team - Torus Lab), với nhiệm vụ:
- Khảo sát các công nghệ Generative AI mới nổi
- Đánh giá tính khả thi và xây dựng các bản mẫu chức năng (Functional Prototypes/PoC)
- Tích hợp các giải pháp vào hệ thống sản phẩm

**Tài nguyên được cung cấp:**
- Hạ tầng GPU: NVIDIA A100 (40GB), T4 (16GB) trên nền tảng đám mây
- Dữ liệu doanh nghiệp thực tế: Tài liệu y tế (CCAM, NGAP), báo cáo tài chính bảo hiểm
- API access: OpenAI GPT-4, Google Gemini, Anthropic Claude



\section{Problem Statement}

During our work in the R\&D department, we identified two core problems that needed to be addressed when applying GenAI in practice, corresponding to two main technical approaches:

\textbf{Problem 1: Integration of External Knowledge from Complex Unstructured Data}

Partner businesses (such as insurance companies, healthcare facilities) possess large amounts of data in the form of PDF documents containing text, tables, and images. Traditional RAG (Retrieval-Augmented Generation) methods based on plain text (text-only) fail to understand the semantics of complex tables or visual information.
The question is: How to build a Multimodal RAG pipeline capable of accurately parsing, indexing, and retrievaling information from these mixed documents to support decision-making (e.g., medical refund code lookup, financial data analysis)?

\textbf{Problem 2: Behavioral Adaptation and Resource Optimization for Small Models (Behavioral Adaptation \& Efficiency)}

For applications requiring high interactivity, counseling, or entertainment (such as psychological counseling chatbots or Tarot Readers), the requirement is not only for information accuracy but also for consistency in tone and style and rule-based reasoning. Using large models (like GPT-4) via APIs is both costly to operate and difficult to fully control behavior.
The question is: Is it possible to fine-tune small language models (such as Qwen, Llama < 7B parameters) using parameter optimization techniques (PEFT/LoRA) and quantization so that they achieve inference capabilities and writing styles comparable to large models, but can be run locally at low cost?

\section{1.3. Phát biểu vấn đề nghiên cứu (Problem Statement)

\subsection{1.3.1. Các câu hỏi nghiên cứu (Research Questions)

Trong quá trình làm việc tại bộ phận R&D, chúng tôi đã xác định hai nhóm vấn đề cốt lõi cần giải quyết khi áp dụng Generative AI vào thực tế, được hình thức hóa thành các câu hỏi nghiên cứu sau:

---

**RQ1 (Research Question 1):** *Làm thế nào để xây dựng một pipeline Multimodal RAG có khả năng phân tích cú pháp (parsing), lập chỉ mục (indexing) và truy xuất (retrieval) chính xác thông tin từ các tài liệu hỗn hợp (văn bản, bảng biểu, hình ảnh) để hỗ trợ ra quyết định trong lĩnh vực y tế và tài chính?*

**Bối cảnh:** Các doanh nghiệp đối tác sở hữu lượng lớn dữ liệu dưới dạng tài liệu PDF chứa văn bản, bảng biểu (tables) và hình ảnh (charts/images). Các phương pháp RAG truyền thống dựa trên văn bản thuần túy (text-only) thất bại trong việc hiểu ngữ nghĩa của bảng biểu phức tạp hoặc thông tin thị giác.

---

**RQ2 (Research Question 2):** *Liệu có thể tinh chỉnh (Fine-tuning) các mô hình ngôn ngữ nhỏ (Small Language Models - SLMs, < 7B tham số) bằng các kỹ thuật tối ưu tham số (PEFT/LoRA) và lượng tử hóa (Quantization) để đạt được khả năng suy luận và văn phong tương đương các mô hình lớn, nhưng có thể chạy cục bộ với chi phí thấp?*

**Bối cảnh:** Đối với các ứng dụng yêu cầu tính tương tác cao, mang tính chất tư vấn (như chatbot tư vấn hoặc hệ thống suy luận dựa trên quy tắc), yêu cầu không chỉ là độ chính xác của thông tin mà còn là sự nhất quán trong văn phong (Tone & Style) và khả năng suy luận theo quy tắc (Rule-based reasoning).

\subsection{1.3.2. Giả thuyết nghiên cứu (Research Hypotheses)

Dựa trên các câu hỏi nghiên cứu, chúng tôi đề xuất các giả thuyết sau:

**H1 (Hypothesis 1):** *Việc kết hợp Hybrid Search (Dense Retrieval + Sparse Retrieval) với Cross-Encoder Reranking và chiến lược Parent-Child Indexing sẽ cải thiện đáng kể độ chính xác truy xuất (Retrieval Accuracy) và độ trung thực (Faithfulness) của hệ thống RAG so với phương pháp Dense Search đơn thuần, đặc biệt trên dữ liệu bảng biểu và mã code y tế.*

**H2 (Hypothesis 2):** *Các mô hình ngôn ngữ nhỏ (1.5B - 3B tham số) sau khi được Fine-tuning bằng QLoRA trên dữ liệu tổng hợp (Synthetic Data) chất lượng cao có thể học được văn phong (Style) và định dạng đầu ra (Output Format) của tác vụ chuyên biệt, nhưng sẽ gặp hạn chế về khả năng suy luận phức tạp (Complex Reasoning) so với các mô hình lớn hơn (>70B tham số).*


\section{Objectives \& Contributions}


\section{Thesis Structure}


%%------------------------------- chapter ------------------------------------

\chapter{Cơ sở lý thuyết}

Chương này cung cấp nền tảng lý thuyết và toán học cần thiết cho các phương pháp được áp dụng trong luận văn. Chúng tôi trình bày có hệ thống từ kiến trúc cơ bản của LLMs đến các kỹ thuật thích ứng tiên tiến, bao gồm Retrieval-Augmented Generation (RAG) và Parameter-Efficient Fine-Tuning (PEFT).

\section{Nền tảng Mô hình Ngôn ngữ Lớn}

\subsection{Kiến trúc Transformer và Cơ chế Self-Attention}

Hầu hết các LLM hiện đại sử dụng kiến trúc \textbf{Decoder-only Transformer} [Vaswani et al., 2017]. Thành phần cốt lõi cho phép mô hình xử lý các phụ thuộc tầm xa (long-range dependencies) là cơ chế \textbf{Self-Attention}.

\subsubsection{Định nghĩa toán học}

Cho một chuỗi đầu vào được biểu diễn bởi ma trận embedding $\mathbf{X} \in \mathbb{R}^{n \times d_{model}}$, trong đó $n$ là độ dài chuỗi và $d_{model}$ là chiều embedding. Cơ chế Self-Attention thực hiện phép biến đổi tuyến tính để tạo ra ba ma trận:

$$\mathbf{Q} = \mathbf{X}\mathbf{W}_Q, \quad \mathbf{K} = \mathbf{X}\mathbf{W}_K, \quad \mathbf{V} = \mathbf{X}\mathbf{W}_V$$

trong đó $\mathbf{W}_Q, \mathbf{W}_K \in \mathbb{R}^{d_{model} \times d_k}$ và $\mathbf{W}_V \in \mathbb{R}^{d_{model} \times d_v}$ là các ma trận trọng số học được.

Đầu ra của Scaled Dot-Product Attention được tính như sau:

$$\text{Attention}(\mathbf{Q}, \mathbf{K}, \mathbf{V}) = \text{softmax}\left(\frac{\mathbf{Q}\mathbf{K}^T}{\sqrt{d_k}}\right)\mathbf{V}$$

\subsubsection{Phân tích vai trò của hệ số tỷ lệ $\frac{1}{\sqrt{d_k}}$}

Hệ số $\frac{1}{\sqrt{d_k}}$ đóng vai trò quan trọng trong việc ổn định số học (numerical stability). Để hiểu tại sao, xét tích vô hướng của hai vector $\mathbf{q}, \mathbf{k} \in \mathbb{R}^{d_k}$ với các thành phần độc lập, có trung bình 0 và phương sai 1:

$$\mathbb{E}[\mathbf{q}^T\mathbf{k}] = 0, \quad \text{Var}[\mathbf{q}^T\mathbf{k}] = d_k$$

Khi $d_k$ lớn (thường $d_k = 64$ hoặc $128$), giá trị $\mathbf{q}^T\mathbf{k}$ có độ lệch chuẩn $\sqrt{d_k}$, đẩy softmax vào vùng bão hòa (saturation regions) với gradient gần 0. Việc chia cho $\sqrt{d_k}$ đưa phương sai về 1, đảm bảo gradient ổn định trong quá trình backpropagation.

\subsubsection{Multi-Head Attention}

Để cho phép mô hình học các biểu diễn từ nhiều không gian con (subspaces) khác nhau, Transformer sử dụng Multi-Head Attention:

$$\text{MultiHead}(\mathbf{Q}, \mathbf{K}, \mathbf{V}) = \text{Concat}(\text{head}_1, \ldots, \text{head}_h)\mathbf{W}_O$$

trong đó $\text{head}_i = \text{Attention}(\mathbf{Q}\mathbf{W}_Q^i, \mathbf{K}\mathbf{W}_K^i, \mathbf{V}\mathbf{W}_V^i)$.

\textbf{Độ phức tạp tính toán:} Self-Attention có độ phức tạp $O(n^2 \cdot d)$ về thời gian và $O(n^2)$ về bộ nhớ, là bottleneck chính khi xử lý chuỗi dài.

\subsection{Quá trình Huấn luyện LLM}

Các LLM hiện đại thường trải qua ba giai đoạn huấn luyện:

\textbf{Giai đoạn 1: Pre-training (Huấn luyện trước)}

Mô hình được huấn luyện trên lượng lớn văn bản không gán nhãn với mục tiêu \textbf{Causal Language Modeling} (dự đoán token tiếp theo):

$$\mathcal{L}_{pretrain} = -\sum_{t=1}^{T} \log P_\theta(x_t | x_{<t})$$

\textbf{Giai đoạn 2: Supervised Fine-Tuning (SFT)}

Mô hình được tinh chỉnh trên dữ liệu có cấu trúc (instruction, response) để học cách tuân theo chỉ dẫn.

\textbf{Giai đoạn 3: Alignment (RLHF/DPO)}

Tối ưu hóa mô hình theo sở thích của con người thông qua Reinforcement Learning from Human Feedback hoặc Direct Preference Optimization.

\subsection{Các Hạn chế Cơ bản của LLMs trong Miền Đặc thù}

Mặc dù có khả năng sinh văn bản mạnh mẽ, các LLM pre-trained gặp ba hạn chế cơ bản khi áp dụng vào miền đặc thù:

\begin{table}[H]
\centering
\begin{tabular}{|l|p{5cm}|p{5cm}|}
\hline
\textbf{Hạn chế} & \textbf{Mô tả} & \textbf{Hệ quả} \\
\hline
\textbf{Hallucination} & Mô hình sinh thông tin không có trong dữ liệu huấn luyện & Không thể tin cậy trong Y tế, Pháp lý \\
\hline
\textbf{Knowledge Cutoff} & Tri thức bị giới hạn bởi thời điểm pre-training & Thiếu thông tin cập nhật \\
\hline
\textbf{Behavioral Rigidity} & Khó thích ứng với văn phong/quy tắc đặc thù & Cần prompt engineering phức tạp \\
\hline
\end{tabular}
\caption{Các hạn chế cơ bản của LLMs}
\end{table}

\textbf{Định lý 2.1 (Giới hạn của In-Context Learning):} \textit{Với mô hình có context window kích thước $C$ tokens, số lượng ví dụ few-shot tối đa có thể đưa vào là $k_{max} = \lfloor C / L_{avg} \rfloor$, trong đó $L_{avg}$ là độ dài trung bình của mỗi ví dụ. Khi tác vụ yêu cầu nhiều quy tắc phức tạp, $k_{max}$ thường không đủ để mô hình học được đầy đủ.}

\section{Retrieval-Augmented Generation (RAG)}

\subsection{Động lực và Tổng quan}

Để giải quyết vấn đề Knowledge Cutoff và Hallucination, Lewis et al. (2020) đề xuất framework \textbf{Retrieval-Augmented Generation (RAG)}, kết nối mô hình sinh văn bản với một cơ chế truy xuất thông tin bên ngoài.

\subsubsection{Định nghĩa hình thức}

Cho $\mathcal{D} = \{d_1, d_2, \ldots, d_N\}$ là kho tài liệu (document corpus) và $q$ là truy vấn của người dùng. Hệ thống RAG thực hiện hai bước:

\textbf{Bước 1 - Retrieval:} Tìm tập $\mathcal{C} \subset \mathcal{D}$ chứa $k$ tài liệu liên quan nhất:
$$\mathcal{C} = \text{Top-}k_{d \in \mathcal{D}} \; \text{sim}(q, d)$$

\textbf{Bước 2 - Generation:} Sinh câu trả lời $a$ dựa trên cả $q$ và $\mathcal{C}$:
$$a = \arg\max_a P_\theta(a | q, \mathcal{C})$$

\subsection{Dense Retrieval và Embedding Vectors}

\subsubsection{Mô hình Bi-Encoder}

Phương pháp Dense Retrieval sử dụng mạng nơ-ron để ánh xạ văn bản vào không gian vector liên tục. Cho encoder $f_\theta: \mathcal{T} \rightarrow \mathbb{R}^d$, vector biểu diễn của truy vấn và tài liệu là:

$$\mathbf{v}_q = f_\theta(q), \quad \mathbf{v}_d = f_\theta(d)$$

\subsubsection{Độ tương đồng Cosine (Cosine Similarity)

Độ liên quan giữa truy vấn và tài liệu được đo bằng Cosine Similarity:

$$\text{sim}_{cos}(\mathbf{v}_q, \mathbf{v}_d) = \frac{\mathbf{v}_q \cdot \mathbf{v}_d}{\|\mathbf{v}_q\| \cdot \|\mathbf{v}_d\|} = \cos(\theta)$$

trong đó $\theta$ là góc giữa hai vector.

**Tính chất toán học:**
- $\text{sim}_{cos} \in [-1, 1]$
- Bất biến với độ dài vector (độ dài văn bản)
- Tập trung vào hướng (semantic orientation) thay vì magnitude

**Mệnh đề 2.1:** *Với các vector đã được chuẩn hóa ($\|\mathbf{v}\| = 1$), Cosine Similarity tương đương với tích vô hướng (Dot Product), và khoảng cách Euclidean có quan hệ đơn điệu:*
$$\|\mathbf{v}_q - \mathbf{v}_d\|^2 = 2(1 - \text{sim}_{cos}(\mathbf{v}_q, \mathbf{v}_d))$$

\subsection{2.2.3. Sparse Retrieval với BM25

\subsubsection{Motivation

Dense Retrieval hiệu quả trong việc bắt ngữ nghĩa (semantic similarity), nhưng có thể thất bại với các truy vấn chứa **từ khóa chính xác** (exact keywords) như mã code y tế (ICD-10, CCAM) hoặc tên riêng.

\subsubsection{Công thức BM25

BM25 (Best Matching 25) là thuật toán Sparse Retrieval dựa trên tần suất từ xuất hiện. Cho truy vấn $Q$ chứa các từ $\{q_1, \ldots, q_n\}$, điểm BM25 của tài liệu $D$ là:

$$\text{BM25}(D, Q) = \sum_{i=1}^{n} \text{IDF}(q_i) \cdot \frac{f(q_i, D) \cdot (k_1 + 1)}{f(q_i, D) + k_1 \cdot \left(1 - b + b \cdot \frac{|D|}{avgdl}\right)}$$

trong đó:
- $f(q_i, D)$: Tần suất xuất hiện của từ $q_i$ trong tài liệu $D$
- $|D|$: Độ dài tài liệu (số từ)
- $avgdl$: Độ dài trung bình của tất cả tài liệu trong corpus
- $k_1, b$: Siêu tham số (thường $k_1 = 1.5$, $b = 0.75$)
- $\text{IDF}(q_i)$: Inverse Document Frequency

$$\text{IDF}(q_i) = \log\left(\frac{N - n(q_i) + 0.5}{n(q_i) + 0.5} + 1\right)$$

với $N$ là tổng số tài liệu và $n(q_i)$ là số tài liệu chứa từ $q_i$.

**Phân tích:** BM25 là hàm **sublinear** theo tần suất từ, tránh việc tài liệu dài quá được ưu tiên.

\subsection{2.2.4. Hybrid Search: Kết hợp Dense và Sparse

\subsubsection{Reciprocal Rank Fusion (RRF)

Để kết hợp kết quả từ hai phương pháp retrieval, chúng tôi sử dụng Reciprocal Rank Fusion [Cormack et al., 2009]:

$$\text{RRF}(d) = \sum_{r \in \mathcal{R}} \frac{1}{k + \text{rank}_r(d)}$$

trong đó $\mathcal{R}$ là tập các danh sách xếp hạng (từ Dense và Sparse), $\text{rank}_r(d)$ là thứ hạng của tài liệu $d$ trong danh sách $r$, và $k$ là hằng số điều chỉnh (thường $k = 60$).

\subsubsection{Linear Combination với Trọng số Thích ứng

Một phương pháp thay thế là kết hợp tuyến tính với trọng số:

$$\text{Score}(q, d) = \alpha \cdot \text{Norm}(S_{dense}(q, d)) + (1 - \alpha) \cdot \text{Norm}(S_{sparse}(q, d))$$

trong đó $\text{Norm}(\cdot)$ là hàm chuẩn hóa Min-Max để đưa điểm số về cùng thang $[0, 1]$.

**Đề xuất của luận văn:** Trọng số $\alpha$ có thể được điều chỉnh động dựa trên đặc trưng của truy vấn $q$:
$$\alpha(q) = \sigma(w^T \cdot \phi(q) + b)$$

trong đó $\phi(q)$ là vector đặc trưng (chứa mã code hay không, độ dài, etc.) và $\sigma$ là hàm sigmoid.

\subsection{2.2.5. Lập chỉ mục hiệu quả với HNSW

\subsubsection{Vấn đề về Độ phức tạp

Với corpus $N$ tài liệu, việc tìm kiếm chính xác (Exact kNN) đòi hỏi tính Cosine Similarity với tất cả $N$ vectors, có độ phức tạp $O(N \cdot d)$. Khi $N$ lên tới hàng triệu, tốc độ truy vấn trở nên không khả thi.

\subsubsection{Thuật toán HNSW (Hierarchical Navigable Small World)

HNSW [Malkov & Yashunin, 2018] xây dựng cấu trúc đồ thị phân tầng cho phép Approximate Nearest Neighbor (ANN) search với độ phức tạp $O(\log N)$.

**Cấu trúc:**
- **Lớp trên (Top layers):** Chứa ít node, các cạnh dài (long-range links) cho phép nhảy nhanh qua không gian vector
- **Lớp dưới (Bottom layers):** Chứa nhiều node hơn, các cạnh ngắn (short-range links) cho độ chính xác cục bộ

**Thuật toán tìm kiếm:**
1. Bắt đầu từ entry point ở lớp cao nhất
2. Greedy search: Di chuyển đến neighbor gần query nhất
3. Khi không còn neighbor tốt hơn, xuống lớp thấp hơn
4. Lặp lại đến lớp 0, trả về $k$ neighbors gần nhất

**Định lý 2.2 (Độ phức tạp HNSW):** *Với cấu hình phù hợp, HNSW đảm bảo độ phức tạp trung bình $O(\log N)$ cho mỗi truy vấn, với recall > 95% so với exact search.*

\subsection{2.2.6. Cross-Encoder Reranking

\subsubsection{Motivation

Bi-Encoder (Dense Retrieval) tính toán embedding độc lập cho query và document, nhanh nhưng thiếu tương tác sâu giữa hai bên. Cross-Encoder khắc phục điều này bằng cách đưa cả cặp $(q, d)$ vào cùng một Transformer.

\subsubsection{Kiến trúc và Công thức

$$\text{Score}_{CE}(q, d) = \text{MLP}(\text{CLS}(\text{BERT}([q; \text{SEP}; d])))$$

trong đó $[q; \text{SEP}; d]$ là chuỗi nối của query và document, và $\text{CLS}$ là hidden state của token [CLS].

**Trade-off:**
- **Ưu điểm:** Độ chính xác cao hơn nhờ self-attention xem xét tương tác giữa mọi cặp token
- **Nhược điểm:** Độ phức tạp $O(N \cdot (|q| + |d|)^2)$, không thể dùng cho toàn bộ corpus

**Quy trình thực tế:**
1. Bi-Encoder retrieve Top-50 ứng viên (nhanh)
2. Cross-Encoder rerank 50 ứng viên này
3. Chọn Top-5 để đưa vào LLM

\subsection{2.2.7. Thách thức với Dữ liệu Đa phương thức

\subsubsection{Vấn đề của PDF phức tạp

Tài liệu thực tế (báo cáo tài chính, danh mục y tế) thường chứa:
- **Văn bản không tuyến tính:** Chia cột, header/footer gây nhiễu
- **Bảng biểu:** Cấu trúc 2D, mất ngữ nghĩa khi flatten thành text 1D
- **Hình ảnh/Biểu đồ:** Không thể đọc bằng text embedding

\subsubsection{Phương pháp xử lý

1. **Document Layout Analysis (DLA):**
   - Sử dụng mô hình Vision (YOLOX, Detectron2) để phát hiện bounding box của các vùng Table, Image, Text
   
2. **Table Preservation:**
   - Chuyển đổi bảng sang Markdown/HTML để giữ cấu trúc hàng-cột
   - LLMs được train trên HTML/Markdown nên có thể "hiểu" cấu trúc 2D

3. **Multimodal Embeddings (CLIP/SigLIP):**
   - Sử dụng mô hình Vision-Language để tạo embedding chung cho text và image
   - Cho phép Text-to-Image retrieval

---

\section{2.3. Tối ưu hóa và Thích ứng Mô hình (Model Optimization and Adaptation)

Trong khi RAG giải quyết vấn đề Knowledge Gap, nó không thay đổi **hành vi nội tại** của mô hình. Phần này trình bày các kỹ thuật để thích ứng LLMs cho các tác vụ yêu cầu văn phong hoặc quy tắc suy luận đặc thù.

\subsection{2.3.1. Supervised Fine-Tuning (SFT)

\subsubsection{Định nghĩa hình thức

Cho tập dữ liệu gán nhãn $\mathcal{D} = \{(x_i, y_i)\}_{i=1}^{N}$, trong đó $x$ là prompt (instruction) và $y$ là response mong muốn. Mục tiêu của SFT là tối ưu hóa:

$$\theta^* = \arg\min_\theta \mathcal{L}_{SFT}(\theta) = \arg\min_\theta \left[ -\sum_{i=1}^{N} \sum_{t=1}^{|y_i|} \log P_\theta(y_i^{(t)} | x_i, y_i^{(<t)}) \right]$$

Đây là **Cross-Entropy Loss** giữa phân phối dự đoán và token thực tế.

\subsubsection{Vấn đề của Full Fine-Tuning

Cập nhật toàn bộ tham số $\theta$ của LLM đối mặt với:
1. **Chi phí tính toán:** Mô hình 7B tham số cần ~28GB VRAM chỉ để lưu gradient (với mixed precision)
2. **Catastrophic Forgetting:** Mô hình có thể "quên" kiến thức tổng quát khi được fine-tune sâu trên tác vụ hẹp
3. **Storage:** Mỗi tác vụ cần lưu một bản copy đầy đủ của mô hình

\subsection{2.3.2. Low-Rank Adaptation (LoRA)

\subsubsection{Động lực và Giả thuyết

Aghajanyan et al. (2021) chứng minh rằng các mô hình ngôn ngữ lớn có **intrinsic dimensionality** thấp, nghĩa là sự thay đổi trọng số cần thiết cho một tác vụ mới nằm trong một không gian con (subspace) chiều thấp.

\subsubsection{Định nghĩa toán học

Cho $\mathbf{W}_0 \in \mathbb{R}^{d \times k}$ là ma trận trọng số pre-trained. Thay vì cập nhật trực tiếp $\mathbf{W}_0$, LoRA [Hu et al., 2021] phân rã sự thay đổi $\Delta \mathbf{W}$ thành tích của hai ma trận rank thấp:

$$\Delta \mathbf{W} = \mathbf{B}\mathbf{A}$$

trong đó $\mathbf{B} \in \mathbb{R}^{d \times r}$, $\mathbf{A} \in \mathbb{R}^{r \times k}$, và $r \ll \min(d, k)$.

\subsubsection{Forward Pass

$$\mathbf{h} = \mathbf{W}_0 \mathbf{x} + \Delta \mathbf{W} \mathbf{x} = \mathbf{W}_0 \mathbf{x} + \mathbf{B}\mathbf{A}\mathbf{x}$$

Trong quá trình training:
- $\mathbf{W}_0$ được **đóng băng** (frozen)
- Chỉ $\mathbf{A}$ và $\mathbf{B}$ được cập nhật

\subsubsection{Phân tích Số lượng Tham số

| Phương pháp | Số tham số trainable | Ví dụ (Llama-7B, d=4096) |
|-------------|---------------------|--------------------------|
| Full Fine-Tuning | $d \times k$ | ~7 tỷ |
| LoRA (r=16) | $r \times (d + k)$ | ~8 triệu |
| **Giảm** | $\frac{r \times (d+k)}{d \times k}$ | **~1000x** |

\subsubsection{Khởi tạo và Scaling

- $\mathbf{A}$: Khởi tạo Gaussian với phương sai nhỏ
- $\mathbf{B}$: Khởi tạo bằng 0 để $\Delta \mathbf{W} = 0$ ban đầu
- Scaling factor: $\alpha / r$ để kiểm soát magnitude của adaptation

$$\mathbf{h} = \mathbf{W}_0 \mathbf{x} + \frac{\alpha}{r} \mathbf{B}\mathbf{A}\mathbf{x}$$

\subsection{2.3.3. Quantization và QLoRA

\subsubsection{Motivation

Để triển khai trên GPU có bộ nhớ hạn chế (như T4 16GB), cần giảm kích thước mô hình thông qua **Quantization** — giảm độ chính xác của trọng số từ 16-bit/32-bit xuống 8-bit hoặc 4-bit.

\subsubsection{Định nghĩa Toán học của Quantization

Quá trình lượng tử hóa ánh xạ giá trị thực $x$ (FP16/FP32) sang giá trị nguyên $x_q$ (INT8/INT4):

$$x_q = \text{round}\left(\frac{x - Z}{S}\right), \quad \hat{x} = S \cdot x_q + Z$$

trong đó:
- $S$ (Scale): Hệ số tỷ lệ
- $Z$ (Zero-point): Điểm không
- $\hat{x}$: Giá trị xấp xỉ sau de-quantization

\subsubsection{NormalFloat 4-bit (NF4)

Dettmers et al. (2023) đề xuất NF4, một phương pháp quantization tối ưu cho trọng số tuân theo phân phối chuẩn:

$$\text{NF4}: q_i = \frac{1}{2}\left(\Phi^{-1}\left(\frac{i}{16}\right) + \Phi^{-1}\left(\frac{i+1}{16}\right)\right), \quad i \in \{0, 1, \ldots, 15\}$$

trong đó $\Phi^{-1}$ là hàm ngược của CDF chuẩn. Các quantization levels được chọn để minimizing expected quantization error.

\subsubsection{QLoRA: Kết hợp Quantization và LoRA

QLoRA [Dettmers et al., 2023] kết hợp:
1. **Mô hình gốc $\mathbf{W}_0$:** Lưu ở dạng 4-bit (NF4)
2. **LoRA adapters $\mathbf{A}, \mathbf{B}$:** Lưu ở dạng 16-bit (BFloat16)

**Lợi ích:**
- Giảm VRAM từ ~28GB (FP16) xuống ~6GB (4-bit) cho mô hình 7B
- Vẫn giữ được độ chính xác gradient nhờ LoRA adapters ở FP16

\subsection{2.3.4. Knowledge Distillation và Synthetic Data Generation

\subsubsection{Motivation

Trong các miền đặc thù (niche domains), dữ liệu huấn luyện chất lượng cao thường khan hiếm. **Knowledge Distillation** cho phép chuyển giao khả năng từ mô hình lớn (Teacher) sang mô hình nhỏ (Student).

\subsubsection{Định nghĩa Toán học

Cho Teacher model $T$ với phân phối đầu ra $P_T(y|x)$ và Student model $S$ với $P_S(y|x)$. Mục tiêu là minimize **Kullback-Leibler Divergence**:

$$\mathcal{L}_{KD} = \mathbb{E}_{x \sim \mathcal{D}} \left[ D_{KL}(P_T(\cdot|x) \| P_S(\cdot|x)) \right]$$

$$D_{KL}(P_T \| P_S) = \sum_y P_T(y|x) \log \frac{P_T(y|x)}{P_S(y|x)}$$

\subsubsection{Quy trình Synthetic Data Generation

1. **Teacher Model** (GPT-4, Gemini): Sinh cặp $(x, y_{teacher})$ dựa trên prompt template
2. **Diversity Injection:** Thêm variation vào input (user personality, edge cases)
3. **Filtering:** Loại bỏ các mẫu có chất lượng thấp
4. **Student Training:** Fine-tune SLM trên dataset này

---

\section{2.4. Phương pháp Đánh giá (Evaluation Methodology)

\subsection{2.4.1. RAGAS Framework

RAGAS [Es et al., 2023] là framework đánh giá RAG không cần ground truth, sử dụng LLM làm judge.

\subsubsection{Faithfulness

Đo lường mức độ câu trả lời được hỗ trợ bởi context:

$$\text{Faithfulness} = \frac{|\{s \in S : s \text{ can be inferred from } C\}|}{|S|}$$

trong đó $S$ là tập các claims trong câu trả lời, $C$ là context.

\subsubsection{Answer Relevance

Đo lường mức độ câu trả lời giải quyết câu hỏi:

$$\text{Relevance} = \frac{1}{n} \sum_{i=1}^{n} \text{sim}_{cos}(\mathbf{v}_q, \mathbf{v}_{q_i})$$

trong đó $q_i$ là câu hỏi được tái tạo từ câu trả lời.

\subsubsection{Context Precision

Đo lường tỷ lệ context có ích trong top-k:

$$\text{Context Precision@K} = \frac{1}{K} \sum_{k=1}^{K} \frac{\sum_{i=1}^{k} v_i}{k} \cdot v_k$$

trong đó $v_k = 1$ nếu chunk tại vị trí $k$ là relevant.

\subsection{2.4.2. Perplexity

Đánh giá độ "bối rối" của language model:

$$\text{PPL} = \exp\left(-\frac{1}{T} \sum_{t=1}^{T} \log P_\theta(x_t | x_{<t})\right)$$

PPL thấp hơn = model tự tin hơn trong việc dự đoán.

\subsection{2.4.3. LLM-as-a-Judge

Sử dụng LLM mạnh (GPT-4) để đánh giá output của LLM yếu hơn trên các tiêu chí định tính:
- Coherence (Mạch lạc)
- Helpfulness (Hữu ích)
- Harmlessness (Vô hại)

---

\section{2.5. Tổng kết Chương

Chương này đã cung cấp nền tảng lý thuyết cho hai phương pháp thích ứng LLM:

| Phương pháp | Giải quyết | Cơ chế | Ưu điểm | Hạn chế |
|-------------|-----------|--------|---------|---------|
| **RAG** | Knowledge Gap | External retrieval | Dễ cập nhật, ít hallucination | Chi phí retrieval, context limit |
| **Fine-tuning (LoRA)** | Behavior Gap | Weight adaptation | Thay đổi hành vi sâu | Cần dữ liệu, có thể overfit |

Các chương tiếp theo sẽ trình bày chi tiết việc triển khai và đánh giá hai phương pháp này trên các use-case thực tế.


%%------------------------------- chapter ------------------------------------

\chapter{Multimodal RAG for Complex Document Understanding}

In this chapter, we present our research, design and implementation of a Multimodal Retrieval-Augmented Generation (RAG) pipeline to address the challenge of extracting and utilizing knowledge from complex unstructured documents containing text, tables, and images. We will detail and analyze each component of the system, explain why the basic RAG approach is insufficient, and demonstrate the techniques we applied to enhance the model's understanding and retrieval capabilities.

Nội dung chương được tổ chức như sau: Phần 3.1 phân tích các hạn chế của RAG truyền thống và đặt ra bài toán nghiên cứu. Phần 3.2 trình bày kiến trúc tổng quan của pipeline đề xuất. Các phần 3.3 đến 3.6 đi sâu vào từng thành phần của pipeline với các lựa chọn thiết kế được giải thích cụ thể. Phần 3.7 và 3.8 trình bày hai use-case thực tế: hệ thống hỏi đáp tài liệu tài chính (GPM) và hệ thống tra cứu mã y tế (Dr. Besnier/CCAM). Cuối cùng, Phần 3.9 trình bày phương pháp đánh giá và kết quả thực nghiệm.

\section{Tại sao RAG Cơ bản Không Đủ? Phân tích Vấn đề}

Trong quá trình làm việc tại Torus AI, chúng tôi đối mặt với các tài liệu doanh nghiệp thực tế từ hai lĩnh vực chính: Y tế (danh mục mã hoàn tiền CCAM - Classification Commune des Actes Médicaux) và Tài chính (báo cáo tài chính của các quỹ bảo hiểm như GPM/AGMF). Những tài liệu này có đặc điểm chung: chúng không phải là văn bản thuần túy mà là sự kết hợp phức tạp của nhiều loại dữ liệu.

Một trang điển hình trong báo cáo tài chính GPM có thể chứa: tiêu đề và văn bản mô tả (chiếm khoảng 30\% nội dung), các bảng số liệu tài chính với nhiều cột và hàng (chiếm khoảng 50\%), và các biểu đồ minh họa xu hướng (chiếm khoảng 20\%). Tương tự, danh mục CCAM chứa hàng nghìn mã y tế được tổ chức theo cấu trúc phân cấp, với các bảng liệt kê mã, mô tả, giá hoàn tiền và các điều kiện áp dụng phức tạp. Chi tiết về  cấu trúc tài liệu này được trình bày trong Phần 3.8. Còn trong phần tiếp theo chúng tôi sẽ phân tích tại sao RAG cơ bản không thể xử lý hiệu quả các tài liệu như vậy nói chung.

\subsection{Thử nghiệm với RAG Cơ bản: Vấn đề Gặp phải}

Để hiểu rõ tại sao cần một giải pháp nâng cao, chúng tôi đã triển khai và đánh giá một pipeline RAG cơ bản (Simple RAG) với các thành phần tiêu chuẩn: sử dụng PyPDF hoặc text extractor cơ bản để trích xuất văn bản, embedding model phổ biến sentence-transformers/all-MiniLM-L6-v2, vector database ChromaDB với Dense Search thuần túy, và LLM Gemma-2-12b-it để sinh câu trả lời.
Kết quả thử nghiệm trên các tài liệu thực tế từ GPM và CCAM cho thấy những hạn chế nghiêm trọng của RAG cơ bản:

\subsubsection{Vấn đề 1: Mất Cấu trúc Bảng biểu (Table Structure Loss)}

Khi trích xuất văn bản từ PDF bằng các công cụ cơ bản, cấu trúc 2D của bảng bị "flatten" thành chuỗi 1D, dẫn đến mất hoàn toàn ngữ nghĩa. Ví dụ, một bảng tài chính với các cột "Năm", "Doanh thu", "Chi phí", "Lợi nhuận" khi được trích xuất sẽ trở thành chuỗi các số và từ khóa không có mối liên hệ rõ ràng. Câu hỏi như "Doanh thu năm 2024 là bao nhiêu?" sẽ không thể trả lời chính xác vì hệ thống không thể mapping giữa năm và giá trị tương ứng.

\subsubsection{Vấn đề 2: Semantic Gap giữa Query và Document (Query-Document Mismatch)}

Embedding model all-MiniLM-L6-v2 được train chủ yếu trên dữ liệu tiếng Anh, trong khi tài liệu của chúng tôi không chỉ chứa tiếng Anh mà phải đa ngôn ngữ. Điều này dẫn đến hiện tượng "semantic gap": các truy vấn tiếng Pháp không được ánh xạ chính xác vào không gian vector tương ứng với tài liệu.

Ngoài ra, một vài tài liệu và các truy vấn chứa mã code cụ thể (như "HAFA008" trong CCAM) gặp khó khăn với Dense Retrieval thuần túy. Embedding model không được train để hiểu rằng "HAFA008" là một entity quan trọng cần match chính xác, dẫn đến việc retrieve các tài liệu có ngữ nghĩa tương tự nhưng không chứa mã code cần tìm.

\subsubsection{Vấn đề 3: Lost in the Middle Phenomenon}

Khi retriever trả về nhiều chunks (ví dụ top-10 hoặc top-20), LLM có xu hướng tập trung vào các chunks ở đầu và cuối context, trong khi bỏ qua thông tin quan trọng nằm ở giữa. Đây là hiện tượng "Lost in the Middle" được Liu et al. (2023) phát hiện. Trong các tài liệu kỹ thuật chứa nhiều mã code tương tự nhau, thông tin chính xác có thể nằm ở chunk thứ 5 hoặc thứ 6, và bị LLM bỏ qua.

\subsection{Định nghĩa Bài toán Nghiên cứu}

Dựa trên phân tích trên, chúng tôi xác định bài toán nghiên cứu: thiết kế một pipeline RAG có khả năng xử lý tài liệu đa phương thức (văn bản, bảng, hình ảnh) với độ chính xác cao cho cả truy vấn ngữ nghĩa lẫn truy vấn chứa keyword/code cụ thể, đồng thời giảm thiểu hiện tượng "Lost in the Middle".

Các yêu cầu cụ thể bao gồm: bảo toàn cấu trúc bảng biểu khi lập chỉ mục, hỗ trợ đa ngôn ngữ (tiếng Pháp là chính), kết hợp khả năng tìm kiếm ngữ nghĩa và keyword matching, và rerank kết quả để đưa thông tin quan trọng nhất lên đầu.

\section{Kiến trúc Tổng quan của Pipeline RAG Nâng cao}

\subsection{Thiết kế Kiến trúc}

Dựa trên phân tích các vấn đề ở Phần 3.1, chúng tôi đề xuất kiến trúc pipeline RAG nâng cao với 7 thành phần chính, được tổ chức thành hai pha: Indexing Phase (offline) và Query Phase (online).

Indexing Phase bao gồm ba bước: Document Parsing sử dụng UnstructuredIO để trích xuất và phân loại nội dung thành Text, Table, Image; Summarization sử dụng LLM để tạo summary cho mỗi loại content; và Vector Storage lưu trữ summary embeddings trong ChromaDB/Qdrant kèm theo liên kết đến raw content trong DocStore.

Query Phase bao gồm bốn bước: Query Processing với optional HyDE transform; Hybrid Retrieval kết hợp Dense Search và Sparse Search (BM25); Reranking sử dụng Cross-Encoder để sắp xếp lại kết quả; và Generation sử dụng LLM với multimodal context để sinh câu trả lời.

\subsection{Lý do Chọn Kiến trúc Này}

Kiến trúc này được thiết kế để giải quyết từng vấn đề đã phân tích. UnstructuredIO giải quyết vấn đề mất cấu trúc bằng cách detect và preserve bảng biểu dưới dạng HTML/Markdown. Summarization tạo "semantic bridge" giữa raw content và query, giúp embedding model hiểu context tốt hơn. Hybrid Search kết hợp sức mạnh của semantic search (Dense) và exact match (Sparse), đặc biệt quan trọng cho queries chứa code. Reranker giải quyết "Lost in the Middle" bằng cách đưa thông tin relevant nhất lên đầu context. Trong các phần tiếp theo, chúng tôi sẽ đi sâu vào từng thành phần của pipeline, giải thích chi tiết các lựa chọn thiết kế và kỹ thuật áp dụng.

\section{Document Parsing: Trích xuất Thông tin Đa phương thức}

\subsection{Lựa chọn Công cụ: Tại sao UnstructuredIO?}

Chúng tôi đánh giá ba công cụ parsing phổ biến: PyPDF/PyMuPDF, LlamaParse, và UnstructuredIO. PyPDF là công cụ cơ bản, nhanh và miễn phí, nhưng chỉ trích xuất text thuần túy, không detect bảng biểu, dẫn đến mất hoàn toàn cấu trúc. LlamaParse là dịch vụ cloud của LlamaIndex, có chất lượng parsing tốt và hỗ trợ nhiều định dạng, nhưng yêu cầu API key, có giới hạn free tier, và phụ thuộc vào kết nối internet. UnstructuredIO là thư viện mã nguồn mở, hỗ trợ Document Layout Analysis với khả năng detect Text, Table, Image regions, bảo toàn cấu trúc bảng dưới dạng HTML, và có thể chạy hoàn toàn local.
Chúng tôi chọn UnstructuredIO vì đáp ứng các yêu cầu: chạy local (không phụ thuộc cloud), bảo toàn cấu trúc bảng, và có thể tùy chỉnh theo nhu cầu.

\subsection{Cấu hình Parsing cho Tài liệu Phức tạp}

Trong pipeline của chúng tôi, UnstructuredIO được cấu hình với các tham số sau: \texttt{strategy=``hi\_res''} để sử dụng Document Layout Analysis chất lượng cao (chậm hơn nhưng chính xác hơn), \texttt{infer\_table\_structure=True} để tự động detect và parse cấu trúc bảng, \texttt{extract\_image\_block\_types=[``Images'', ``Tables'']} để trích xuất hình ảnh và bảng dưới dạng base64, \texttt{chunking\_strategy=``by\_title''} để chunk theo tiêu đề section thay vì fixed-size, \texttt{max\_characters=10000} cho độ dài tối đa mỗi chunk, và \texttt{combine\_text\_under\_n\_chars=500} để gộp các đoạn văn ngắn.

\subsection{Xử lý Từng Loại Content}

Xử lý Văn bản (Text): Các đoạn văn bản được trích xuất và gộp thành CompositeElement dựa trên cấu trúc tiêu đề. Điều này đảm bảo mỗi chunk chứa một đơn vị ngữ nghĩa hoàn chỉnh (ví dụ: một section) thay vì bị cắt giữa chừng.
Xử lý Bảng biểu (Tables): Đây là điểm khác biệt quan trọng nhất so với RAG cơ bản. UnstructuredIO trả về bảng dưới dạng HTML trong thuộc tính \texttt{metadata.text\_as\_html}. Chúng tôi bảo toàn định dạng này vì LLMs được train trên lượng lớn HTML/Markdown nên có khả năng ``hiểu'' cấu trúc 2D. Việc chuyển đổi bảng thành Markdown hoặc giữ nguyên HTML giúp LLM phân biệt được header, row, và cell, từ đó trả lời chính xác các câu hỏi như ``Doanh thu năm 2024 là bao nhiêu?''.

Xử lý Hình ảnh (Images): Hình ảnh được trích xuất dưới dạng base64-encoded string và lưu trong DocStore. Để hỗ trợ retrieval, chúng tôi sử dụng Vision-Language Model (Gemma-3-12b-it có khả năng multimodal) để tạo mô tả chi tiết cho mỗi hình ảnh. Mô tả này được dùng để tạo embedding, cho phép text query có thể retrieve được hình ảnh liên quan.
Xử lý Hình ảnh (Images): Hình ảnh được trích xuất dưới dạng base64-encoded string và lưu trong DocStore. Để hỗ trợ retrieval, chúng tôi sử dụng Vision-Language Model (Gemma-3-12b-it có khả năng multimodal) để tạo mô tả chi tiết cho mỗi hình ảnh. Mô tả này được dùng để tạo embedding, cho phép text query có thể retrieve được hình ảnh liên quan.

\section{Summarization: Tạo Cầu nối Ngữ nghĩa}
\subsection{Tại sao Cần Summarization?}
Một câu hỏi tự nhiên là: tại sao không embed trực tiếp raw content mà cần qua bước summarization? Có ba lý do chính.
Thứ nhất, raw content thường dài và chứa nhiều noise. Một bảng tài chính có thể chứa hàng chục số liệu, nhưng query chỉ hỏi về một giá trị cụ thể. Summary giúp ``nén'' thông tin quan trọng, tăng signal-to-noise ratio.
Thứ hai, summary tạo ``semantic bridge'' giữa query và raw content. Ví dụ, một bảng chỉ chứa số liệu ``2024: 1,234,567 EUR'' mà không có context. Summary có thể là ``Doanh thu năm 2024 của công ty là 1,234,567 EUR, tăng 15\% so với năm trước'', giúp embedding model hiểu ngữ cảnh tốt hơn.
Thứ ba, cho Image retrieval: hình ảnh không thể embed trực tiếp bằng text embedding model. Summary mô tả nội dung hình ảnh cho phép text-to-image retrieval.
\subsection{Chiến lược Summarization}
Chúng tôi sử dụng LLM Gemma-3-12b-it (được host bởi Torus Lab thông qua GENTL API) để tạo summary. Prompt được thiết kế với các yêu cầu cụ thể: tóm tắt chi tiết nội dung bảng hoặc văn bản, bao gồm các thuật ngữ quan trọng của lĩnh vực trong summary, trả lời bằng ngôn ngữ của tài liệu (tiếng Pháp), và chỉ trả về summary mà không có comment thêm.
Prompt này được thiết kế để đảm bảo summary: giữ lại domain-specific terms (quan trọng cho keyword matching), giữ nguyên ngôn ngữ gốc (tránh translation bias), và ngắn gọn (dễ embed).
\subsection{Chiến lược Parent-Child Indexing}
Một innovation quan trọng trong pipeline của chúng tôi là Parent-Child Indexing: chúng tôi lưu summary vào VectorStore (để tìm kiếm) nhưng liên kết đến raw content trong DocStore (để đưa vào LLM).
Quy trình hoạt động như sau: Khi người dùng hỏi một câu hỏi, hệ thống tìm summary tương tự nhất trong VectorStore. Sau đó, thông qua ID liên kết, hệ thống lấy raw content tương ứng từ DocStore. Cuối cùng, raw content (bao gồm cả bảng HTML và hình ảnh base64) được đưa vào LLM để sinh câu trả lời.
Lợi ích của phương pháp này là rõ ràng: summary ngắn gọn giúp retrieval chính xác hơn, trong khi raw content đầy đủ giúp LLM có đủ thông tin để trả lời chi tiết.

\section{Embedding Model: Lựa chọn và Lý do}
\subsection{Đánh giá Các Lựa chọn}

Việc chọn embedding model phù hợp là critical cho hiệu suất RAG. Chúng tôi đánh giá ba lựa chọn chính dựa trên benchmark MTEB (Massive Text Embedding Benchmark).

sentence-transformers/all-MiniLM-L6-v2 là model mặc định trong nhiều tutorial, với kích thước nhỏ (22M params) và tốc độ nhanh. Tuy nhiên, model này được train chủ yếu trên tiếng Anh, hiệu suất kém trên tiếng Pháp, với MTEB French Score khoảng 0.45.

intfloat/multilingual-e5-large là model đa ngôn ngữ với 560M params, có hiệu suất tốt trên nhiều ngôn ngữ bao gồm tiếng Pháp, với MTEB French Score khoảng 0.65. Model này support instruction-following (prefix "query:" và "passage:").

intfloat/multilingual-e5-large-instruct là phiên bản instruction-tuned của E5-large, có hiệu suất tốt nhất trong các model dưới 1B params trên benchmark tiếng Pháp, với MTEB French Score khoảng 0.68. Model này cho phép customize task description.

\subsection{Lựa chọn và Justification}

Chúng tôi chọn multilingual-e5-large-instruct cho Advanced RAG pipeline vì những lý do sau:
Hiệu suất tiếng Pháp tốt nhất: Với đa số tài liệu là tiếng Pháp (báo cáo tài chính, danh mục CCAM), việc sử dụng model có hiệu suất cao trên ngôn ngữ này là quan trọng. E5-large-instruct đạt điểm cao nhất trong benchmark MTEB French trong số các model dưới 1B params.

Instruction-following: Model cho phép prefix query với task description cụ thể, giúp fine-tune retrieval behavior theo context. Ví dụ, với task description "Given a query, retrieve relevant passages that answer the query", model hiểu rằng mục tiêu là tìm passages chứa câu trả lời, không chỉ là passages tương tự về ngữ nghĩa.

Cân bằng giữa hiệu suất và tài nguyên: Với 560M params, model đủ mạnh để capture semantic nuances nhưng vẫn có thể chạy trên GPU 16GB (NVIDIA T4).
Đối với Simple RAG (baseline), chúng tôi sử dụng all-MiniLM-L6-v2 để có điểm so sánh công bằng với các implementation phổ biến.

\section{Hybrid Search và Reranking: Kết hợp Semantic và Keyword}
\subsection{Vấn đề của Dense Search Thuần túy}

Dense Search (vector similarity search) rất hiệu quả trong việc tìm kiếm ngữ nghĩa: câu hỏi "Doanh thu công ty là bao nhiêu?" có thể match với passage chứa "revenue" hoặc "chiffre d'affaires" nhờ semantic similarity. Tuy nhiên, Dense Search gặp khó khăn với các truy vấn chứa exact keywords hoặc codes.
Ví dụ, trong use-case CCAM, bác sĩ có thể hỏi: "Mã HAFA008 dùng cho trường hợp nào?". Embedding của "HAFA008" không có semantic meaning rõ ràng vì đây là một mã code, không phải từ ngữ tự nhiên. Dense Search có thể trả về các passages chứa các mã tương tự (HAFA007, HAFA009) nhưng không chứa đúng HAFA008.

\subsection{Giải pháp: Hybrid Search}
Hybrid Search kết hợp hai phương pháp retrieval: Dense Search cho semantic similarity và Sparse Search (BM25) cho exact keyword matching.
Dense Search sử dụng embedding model để ánh xạ query và documents vào không gian vector, sau đó tìm documents có cosine similarity cao nhất với query. Phương pháp này tốt cho các truy vấn ngữ nghĩa tự nhiên nhưng kém với exact match.
Sparse Search (BM25) sử dụng term frequency và inverse document frequency để tính relevance score. Phương pháp này không hiểu ngữ nghĩa (không biết "doanh thu" và "revenue" liên quan) nhưng rất tốt cho exact keyword matching. Nếu query chứa "HAFA008", BM25 sẽ ưu tiên documents chứa chính xác chuỗi ký tự này.
Trong implementation của chúng tôi sử dụng Qdrant Vector Store, Hybrid Search được kích hoạt với cấu hình \texttt{enable\_hybrid=True} và \texttt{fastembed\_sparse\_model="Qdrant/bm25"}. Kết quả từ hai phương pháp được kết hợp bằng Reciprocal Rank Fusion (RRF).

\subsection{Cross-Encoder Reranking: Giải quyết ``Lost in the Middle''}

Sau khi Hybrid Search trả về top-k candidates (thường k=30-50), chúng tôi sử dụng Cross-Encoder để rerank và chọn ra top-n (thường n=5-10) để đưa vào LLM.
Tại sao cần Reranking? Bi-Encoder (Dense Search) tính embedding độc lập cho query và document, nhanh nhưng không capture được fine-grained interactions giữa hai bên. Cross-Encoder đưa cả cặp (query, document) vào cùng một Transformer, cho phép attention giữa mọi cặp token, từ đó đánh giá relevance chính xác hơn.
Lựa chọn Reranker: Chúng tôi sử dụng cross-encoder/ms-marco-MiniLM-L-12-v2 vì đây là model được train trên MS MARCO dataset, một benchmark phổ biến cho passage ranking, với kích thước nhỏ (33M params) cho phép inference nhanh, và đạt hiệu suất tốt trong nhiều evaluations.
Quy trình: Bi-Encoder retrieve top-50 candidates (nhanh, O(log N) với HNSW index), Cross-Encoder score 50 candidates này (chậm hơn nhưng chỉ 50 documents), và chọn top-5 có score cao nhất để đưa vào LLM.

\subsection{Tác động của Reranking}
Reranking giải quyết "Lost in the Middle" bằng cách đảm bảo thông tin relevant nhất nằm ở đầu context. Trong các thử nghiệm của chúng tôi, nhiều trường hợp thông tin chính xác nằm ở vị trí 5-10 trong kết quả Dense Search, nhưng sau khi reranking được đưa lên vị trí 1-3.


\section{Use-Case 1: Hệ thống Hỏi đáp Tài liệu Tài chính (GPM)}
\subsection{Mô tả Bài toán}

GPM (Gestion Patrimoine Mutualiste) là một use-case từ đối tác của Torus AI, liên quan đến các tài liệu tài chính của quỹ bảo hiểm tương hỗ AGMF (Association Générale des Médecins de France). Các tài liệu bao gồm báo cáo tài chính hàng năm (Comptes), báo cáo Solvency II (SFCR - Solvency and Financial Condition Report), và các tài liệu kèm theo.
Đặc điểm tài liệu: Các tài liệu PDF có độ dài 50-200 trang, chứa nhiều bảng số liệu tài chính phức tạp (bảng cân đối kế toán, báo cáo kết quả kinh doanh), biểu đồ xu hướng, và văn bản phân tích. Ngôn ngữ chủ đạo là tiếng Pháp.
Loại câu hỏi: Người dùng cần tra cứu các thông tin như "Chiffre d'affaires de l'année 2024 est à combien?" (Doanh thu năm 2024 là bao nhiêu?), "Quel est le ratio de solvabilité?" (Tỷ lệ khả năng thanh toán là gì?), hoặc "Expliquez l'évolution des actifs incorporels" (Giải thích sự biến động của tài sản vô hình).

\subsection{Triển khai Pipeline cho GPM}
Dựa trên kiến trúc đề xuất ở Phần 3.2, chúng tôi triển khai pipeline cụ thể cho GPM sử dụng framework LangChain với các thành phần sau:
\begin{itemize}
    \item \textbf{Document Parsing:} UnstructuredIO với partition\_pdf(), cấu hình strategy="hi\_res", infer\_table\_structure=True, và chunking theo title. Kết quả parsing được phân loại thành Texts (CompositeElement), Tables (với HTML trong metadata), và Images (base64).
    \item \textbf{Summarization:} Sử dụng Gemma-3-12b-it qua GENTL API. Texts và Tables được summarize bằng text prompt, trong khi Images được summarize bằng vision prompt yêu cầu mô tả chi tiết biểu đồ.
    \item \textbf{Vector Store:} ChromaDB với MultiVectorRetriever pattern. Summaries được embed và lưu vào ChromaDB, trong khi raw content được lưu vào InMemoryStore với liên kết qua doc\_id.
    \item \textbf{Retrieval:} Dense Search (do tài liệu chủ yếu là văn bản và số liệu, không có nhiều exact codes cần match).
    \item \textbf{Generation:} Gemma-3-12b-it với multimodal prompt hỗ trợ cả text context và images (base64). Prompt yêu cầu model trả lời dựa trên context được cung cấp, bằng ngôn ngữ của tài liệu.
\end{itemize}

\subsection{Ví dụ Minh họa}
Giả sử người dùng hỏi: ``Combien est le chiffre d'affaires de l'année 2024?'' (Doanh thu năm 2024 là bao nhiêu?).
\begin{enumerate}
    \item \textbf{Retrieval:} Query được embed và tìm kiếm trong ChromaDB. Top summaries có thể bao gồm ``Ce tableau présente les résultats financiers de l'entreprise de 2022 à 2024, montrant une augmentation du chiffre d'affaires...'' và ``Le graphique montre l'évolution des revenus sur les 5 dernières années...''.
    \item \textbf{Fetch Raw Content:} Dựa trên doc\_ids của summaries, hệ thống lấy raw content từ DocStore, bao gồm bảng HTML với số liệu chi tiết theo năm.
\end{enumerate}

\section{Use-Case 2: Hệ thống Tra cứu Mã Y tế (Dr. Besnier / CCAM)}
\subsection{Mô tả Bài toán}
Dr. Besnier là một use-case từ khách hàng y tế, liên quan đến việc tra cứu mã hoàn tiền trong danh mục CCAM (Classification Commune des Actes Médicaux). CCAM là hệ thống mã hóa các hành vi y tế tại Pháp, được sử dụng để xác định mức hoàn tiền bảo hiểm.

Đặc điểm dữ liệu: Tài liệu CCAM (file Excel/PDF) chứa hàng nghìn mã y tế, mỗi mã có nhiều thuộc tính: Code (ví dụ: HAFA008), Texte (mô tả hành vi), Tarif Secteur 1 (giá sector 1), Tarif Hors Secteur (giá ngoài sector), và các Conditions (điều kiện áp dụng phức tạp).

Loại câu hỏi: Bác sĩ cần tra cứu như ``Patiente 69 ans, exérèse de carcinome basocellulaire de la lèvre, suture par lambeau à la volée'' (Bệnh nhân 69 tuổi, cắt bỏ ung thư biểu mô đáy môi, khâu bằng vạt da). Hệ thống cần trả về các mã CCAM phù hợp (ví dụ: HAFA008, QAMA002) cùng với thông tin chi tiết.

\subsection{Thách thức Đặc thù}
Use-case này có những thách thức riêng so với GPM:

\begin{itemize}
    \item \textbf{Exact Code Matching:} Người dùng có thể hỏi trực tiếp về một mã cụ thể. Dense Search thuần túy không đảm bảo retrieve đúng mã.
    \item \textbf{Medical Terminology:} Tài liệu chứa nhiều thuật ngữ y tế chuyên môn bằng tiếng Pháp.
    \item \textbf{Cấu trúc Dữ liệu Phức tạp:} Mỗi mã có nhiều thuộc tính liên quan, và điều kiện áp dụng có thể phụ thuộc vào ngữ cảnh bệnh nhân (tuổi, bệnh lý kèm theo).
\end{itemize}

\subsection{Triển khai Pipeline cho CCAM}
Dựa trên các thách thức trên, chúng tôi điều chỉnh pipeline sử dụng framework LlamaIndex:

\begin{itemize}
    \item \textbf{Document Parsing:} UnstructuredIO với partition() để handle cả Excel và PDF. Dữ liệu được parse thành documents, sau đó chunk bằng SentenceSplitter với chunk\_size=256 và chunk\_overlap=32.
    \item \textbf{Embedding Model:} intfloat/multilingual-e5-large cho hỗ trợ tiếng Pháp tốt hơn.
    \item \textbf{Vector Store:} QdrantVectorStore với in-memory mode, cấu hình enable\_hybrid=True và fastembed\_sparse\_model="Qdrant/bm25" để kích hoạt Hybrid Search. HNSW được cấu hình với m=16 và ef\_construct=100 cho cân bằng giữa recall và tốc độ.
    \item \textbf{Retrieval:} Hybrid Search với similarity\_top\_k=30 (Dense) và sparse\_top\_k=10 (BM25).
    \item \textbf{Reranking:} SentenceTransformerRerank với model cross-encoder/ms-marco-MiniLM-L-12-v2, chọn top\_n=20 để đưa vào LLM.
    \item \textbf{Query Transform (Optional):} HyDEQueryTransform có thể được sử dụng để expand query trước khi retrieval, giúp cải thiện recall cho các truy vấn ngắn.
\end{itemize}

\subsection{Ví dụ Minh họa}
Giả sử bác sĩ nhập: ``Patiente 69 ans exérèse de carcinome basocellulaire de la lèvre, suture par lambeau à la volée''.

\begin{enumerate}
    \item \textbf{Query Processing:} Query được format với instruction prefix để embedding model hiểu đây là medical retrieval task.
    \item \textbf{Hybrid Retrieval:} Dense Search tìm passages liên quan đến ``carcinome'', ``lèvre'', ``lambeau'' dựa trên ngữ nghĩa. Đồng thời, BM25 tìm passages chứa các từ khóa y tế cụ thể. Kết quả được merge bằng RRF.
    \item \textbf{Reranking:} Cross-Encoder đánh giá lại relevance của 30 candidates, đưa các passages chứa thông tin về cắt bỏ ung thư môi và kỹ thuật vạt da lên đầu.
    \item \textbf{Generation:} LLM nhận context và trả về các mã phù hợp như ``HAFA008 - Exérèse de lésion de la lèvre...'' và ``QAMA002 - Réparation de perte de substance par lambeau...''.
\end{enumerate}

\section{Đánh giá Thực nghiệm và Kết quả}
\subsection{Phương pháp Đánh giá}
Để đánh giá hiệu quả của pipeline, chúng tôi sử dụng kết hợp hai phương pháp: \textbf{LLM-as-a-Judge} và \textbf{RAGAS Framework}.

\begin{itemize}
    \item \textbf{LLM-as-a-Judge:} Sử dụng LLM mạnh làm evaluator để đánh giá chất lượng câu trả lời. Chúng tôi sử dụng hai models làm judge: Gemma-3-12b-it và Mistral-Small-24B-Instruct-2501 để đảm bảo độ tin cậy. Prompt yêu cầu judge đánh giá hai metrics trên thang 0--1: \textit{Relevancy} (mức độ câu trả lời giải quyết câu hỏi) và \textit{Precision} (mức độ chính xác dựa trên context).
    \item \textbf{RAGAS Framework:} Sử dụng các metrics tiêu chuẩn: \textit{Faithfulness} (câu trả lời có được support bởi context không), \textit{Answer Correctness} (so sánh với ground truth), \textit{Context Recall} (context có chứa thông tin cần thiết không), và \textit{Context Precision} (thông tin relevant có ở đầu context không).
\end{itemize}

\subsection{Dataset Đánh giá}
Chúng tôi xây dựng bộ test cho mỗi use-case. Đối với GPM, chúng tôi tạo \texttt{evals\_set\_copy.jsonl} chứa 25 cặp (question, context, answer) được human-annotated, bao gồm các câu hỏi về số liệu tài chính, xu hướng, và giải thích. Đối với CCAM, chúng tôi xây dựng \texttt{eval\_set\_v1.json} chứa các consultation scenarios với ground truth codes, được verify bởi domain expert (bác sĩ).

\subsection{Kết quả So sánh Simple RAG vs Advanced RAG}
Bảng dưới đây trình bày kết quả đánh giá trên use-case GPM:

\begin{table}[H]
    \centering
    \begin{tabular}{|l|l|c|c|}
        \hline
        \textbf{Pipeline} & \textbf{Judge} & \textbf{Precision} & \textbf{Relevancy} \\
        \hline
        Simple RAG & Gemma-3-12b-it & 0.728 & 0.828 \\
        Advanced RAG & Mistral-Small-24B & 0.976 & 0.972 \\
        \hline
    \end{tabular}
    \caption{So sánh kết quả giữa Simple RAG và Advanced RAG trên use-case GPM}
\end{table}

Kết quả cho thấy Advanced RAG cải thiện đáng kể so với Simple RAG: Precision tăng từ 0.728 lên 0.976 (tăng 34\%) và Relevancy tăng từ 0.828 lên 0.972 (tăng 17.4\%).

\subsection{Phân tích Chi tiết với RAGAS Metrics}
Khi sử dụng RAGAS framework đánh giá trên một subset, chúng tôi thu được kết quả chi tiết hơn. Với \textit{Single-hop queries} (câu hỏi trực tiếp, một bước suy luận), hệ thống đạt Faithfulness 1.0 và Context Precision 1.0, cho thấy khả năng retrieve và answer chính xác. Với \textit{Multi-hop abstract queries} (câu hỏi yêu cầu tổng hợp), hệ thống đạt Faithfulness 0.833 và Answer Correctness 0.810. Với \textit{Multi-hop specific queries} (câu hỏi cụ thể yêu cầu nhiều bước), Context Precision giảm xuống 0.0 trong một số trường hợp, cho thấy thách thức với complex reasoning.

\subsection{Phân tích Nguyên nhân Cải thiện}
Sự cải thiện của Advanced RAG đến từ ba yếu tố chính:
\begin{itemize}
    \item \textbf{Embedding Model tốt hơn:} Multilingual-E5-large-instruct capture ngữ nghĩa tiếng Pháp tốt hơn all-MiniLM-L6-v2, dẫn đến retrieval accuracy cao hơn.
    \item \textbf{Hybrid Search:} Kết hợp Dense và Sparse search giúp handle cả semantic queries lẫn keyword-based queries. Đặc biệt hiệu quả cho CCAM use-case với các mã code cụ thể.
    \item \textbf{Reranker:} Cross-Encoder giải quyết ``Lost in the Middle'', đảm bảo thông tin relevant nhất nằm ở đầu context, giúp LLM tập trung vào thông tin đúng.
\end{itemize}

\subsection{Giám sát và Quan sát (Observability)}
Để giám sát pipeline trong production, chúng tôi khảo sát và tích hợp các công cụ observability: Arize Phoenix cho tracing và embedding visualization, Opik cho evaluation tracking, và LangFuse cho LLM monitoring. Các công cụ này cho phép theo dõi latency từng component, visualize embedding distributions để phát hiện drift, và log evaluation metrics theo thời gian.

\section{Thảo luận và Hạn chế}
\subsection{Thảo luận về Kết quả}
Kết quả thực nghiệm xác nhận giả thuyết H1 đặt ra ở Chương 1: việc kết hợp Hybrid Search với Reranking và chiến lược xử lý tài liệu đa phương thức cải thiện đáng kể hiệu suất RAG so với baseline. Cụ thể, sự cải thiện đến từ việc giải quyết ba vấn đề chính: mất cấu trúc bảng biểu được giải quyết bằng UnstructuredIO và HTML/Markdown preservation, semantic gap được giải quyết bằng multilingual embedding model, và ``Lost in the Middle'' được giải quyết bằng Cross-Encoder reranking.

\subsection{Hạn chế của Nghiên cứu}
Nghiên cứu còn một số hạn chế cần được acknowledge:
\begin{itemize}
    \item \textbf{Dataset size:} Bộ test chỉ có 25 câu hỏi cho GPM và khoảng 10 scenarios cho CCAM. Cần mở rộng để đánh giá toàn diện hơn.
    \item \textbf{Chưa có Ablation Study đầy đủ:} Chúng tôi chưa đánh giá riêng lẻ contribution của từng component (chỉ so sánh Simple vs Advanced). Một ablation study chi tiết sẽ giúp hiểu rõ hơn tầm quan trọng của mỗi thành phần.
    \item \textbf{Latency và Cost:} Chưa phân tích chi tiết trade-off giữa accuracy và latency/cost. Reranking và summarization tăng latency đáng kể.
    \item \textbf{Generalizability:} Pipeline được thiết kế cho tiếng Pháp và các domain cụ thể (y tế, tài chính). Cần đánh giá trên các ngôn ngữ và domain khác.
\end{itemize}

\subsection{Hướng Phát triển}
Dựa trên kết quả và hạn chế, chúng tôi đề xuất các hướng phát triển sau:
\begin{itemize}
    \item \textbf{Mở rộng Dataset và Ablation Study:} Xây dựng bộ test lớn hơn (100+ câu hỏi mỗi use-case) và thực hiện ablation study đầy đủ để đánh giá contribution của từng component.
    \item \textbf{Adaptive Hybrid Search:} Thay vì sử dụng trọng số cố định cho Dense và Sparse, xây dựng classifier để tự động điều chỉnh trọng số dựa trên đặc điểm query (chứa code hay không, độ dài, domain).
    \item \textbf{Query Routing:} Sử dụng small classifier để route queries đến different retrieval strategies (dense-only cho semantic queries, sparse-heavy cho keyword queries).
    \item \textbf{Caching và Optimization:} Implement caching cho summaries và embeddings để giảm latency. Explore model distillation cho reranker.
\end{itemize}

\section{Kết luận Chương}
Chương này đã trình bày chi tiết quá trình thiết kế, triển khai và đánh giá một pipeline RAG nâng cao cho tài liệu đa phương thức. Xuất phát từ phân tích các hạn chế của RAG cơ bản (mất cấu trúc bảng, semantic gap, Lost in the Middle), chúng tôi đề xuất kiến trúc kết hợp nhiều kỹ thuật tiên tiến: UnstructuredIO cho parsing đa phương thức, Summarization với LLM cho semantic bridging, Multilingual embedding cho hỗ trợ tiếng Pháp, Hybrid Search cho kết hợp semantic và keyword matching, và Cross-Encoder Reranking cho prioritization.

Kết quả thực nghiệm trên hai use-cases (GPM tài chính và CCAM y tế) cho thấy sự cải thiện đáng kể: Precision tăng từ 0.728 lên 0.976 và Relevancy tăng từ 0.828 lên 0.972. Những kết quả này xác nhận tính hiệu quả của phương pháp đề xuất và đóng góp thực tiễn cho việc triển khai RAG trong môi trường doanh nghiệp.

Chương tiếp theo sẽ trình bày phương pháp thứ hai để thích ứng LLM: Fine-tuning các mô hình ngôn ngữ nhỏ để học hành vi và văn phong đặc thù, một hướng tiếp cận bổ sung cho RAG khi yêu cầu không chỉ là độ chính xác thông tin mà còn là sự nhất quán trong cách thức trả lời.

\end{document}
